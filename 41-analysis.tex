\subsection{Analysis of OHD-GAN}
\subsubsection{Data representation and algorithm architecture}
We observed that majority of methods included in the review made use of  altered representations of patient records. Namely, through feature engineering the data is transformed from its original form. This is in part due to the inconvenient properties of health data, such as missingess. However, it is somewhat apparent that the main motive is to accommodate existing algorithms. Along with demographic variables, OHD data mostly takes the form of triples composed by a timestamp, a medical concept and the recorded value. Their count is different for each patient, irregular intervals between each triple and the number of possible values in a dimensions can be huge. Moreover, there are generally multiple episodes of care, each with a different cause. The form and content is not typically considered practical for machine learning. \par
At varying degrees, depending on the transformations, information is being lost or bias is being  introduced. For example, when data are reduced by aggregation to one-hot encoding of binary or count variables, the complex relationships found in medical data are, for the most part, lost. Similarly, information is lost when forcing continuous time-series into a regular representation, by truncating, padding, binning or imputation. Moreover, it is highly unlikely that the data is missing at random, introducing the potential for bias when a large part of the real data is rejected on this basis. Truncating the medical codes to their parent generalizations \cite{Zhang2020, choi2017generating}.  In brief, loss of information content is being preferred by molding and discarding arbitrarily the data to the benefit of performance metrics, as opposed to the more tricky alternative of developing algorithms according the data.\par
Deep architectures are based on the intuition that multiple layers of nonlinear functions are needed to learn complicated high-level abstractions \cite{Bengio_2009}. CNN capture patterns of an image in a hierarchical fashion, such that in sequence, each layer forms a representation the data at a higher level of abstraction. This type of data-oriented architecture has led to impressive performance for CNN and image data. The same principle can be applied to health data. An algorithm developed in a hierarchical structure, was demonstrated to form representations of EHR that capture the sequential order of visits and co-occurrence of codes in a visit have led to improved predictor performance, and also allowed for meaningful interpretation of the model \cite{choi2016multi}. Similarly, models of time-series based on a continuous time representation, such as found in EHR data, have shown improved accuracy over discrete time-representations \cite{rubanova2019latent,de2019gru}. Nonetheless, creative adaptations of the data for existing architectures have provided surprising results. For example, OHD input into a CNN were transformed to image(bitmaps) in which the pixels encoded the information \cite{Fukae2020}.

\section{Recommendations}\label{sec:recommend}
\subsection{Basic models}\label{sec:basic}

Overall, evaluation methods were superficial or unidimensional. Finding convincing and robust evaluation metrics for synthetic health data is an open issue. Even more so when the learning task is poorly defined or the scope of the problem is too large. The difficulty of explaining or validating the realism of data representing a patient, often longitudinal and which factors differentially contribute to disease characterization makes the assessment of synthetic data ambiguous, thus demanding stronger evidence to claims.\par
Modelling efforts for OHD-GAN should be limited in scope to a single data type or modality. This is favourable for a number of evaluation related aspects. Firstly, it makes qualitative evaluation by visual inspection from experts possible and meaningful. Secondly, for same reasons, the behaviour of the model can be assessed straightforwardly. The generative process can be influenced intentionally to observe the effect on the properties of the output. Finally, it allows for quantitative evaluation with domain specific metrics. The scope should clearly identify the purpose of the data generation, its utility and the target patients\cite{Capobianco2020,Kappen_2016, Kappen_2016a}

\subsection{Data-driven architecture}\label{sec:archi}
The algorithm architecture of OHD-GAN should be engineered to match the process that generated the data, not the other way around. Data should be used and generated in the form it is first collected. In addition to preventing information loss, this ensures models will reflect the real generative process. Such models are more likely to provide insights into the system they are taught to imitate and further our understanding about them. Furthermore, the learned statistical distribution is inevitably more meaningful and interpretable, facilitating applications in the healthcare domain and supporting the inference of insights from the learned model parameters.
\subsection{Interpretability}
Even though a few authors explored the behavior of their models according to various methods, the subject was left largely unmentioned. It is imperative that future experimentation and publication give equal importance to evaluating the interpretation of their models and means to do so, as for performance. In the healthcare domain, black box machine learning models find little adoption, and synthetic data is most often met with attacks to its validity.