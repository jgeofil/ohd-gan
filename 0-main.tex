\documentclass[9pt,acronym]{article}

\usepackage{fullpage, setspace, parskip, titlesec}
\usepackage[section]{placeins}
\usepackage[dvipsnames]{xcolor}
\usepackage{breakcites, lineno, hyphenat, rotating, etoolbox}
\usepackage[colorlinks]{hyperref}
\usepackage{glossaries, glossaries-extra}
\usepackage[round]{natbib}
\usepackage{authblk, graphicx}
\usepackage[space]{grffile}
\usepackage{latexsym, textcomp, longtable, tabulary, tabularx}
\usepackage{booktabs,array,multirow}
\usepackage{amsfonts,amsmath,amssymb}
\usepackage[utf8]{inputenc}
\usepackage[greek,english]{babel}
\usepackage{xspace, caption}
\usepackage[ampersand]{easylist}
\usepackage{enumitem}
\usepackage{pifont, etex, makecell}
\usepackage[]{todonotes}
\usepackage{float}
\usepackage{lscape}

\hypersetup{
    colorlinks = true,
    linkcolor = blue,
    anchorcolor = blue,
    citecolor = teal,
    filecolor = blue,
    urlcolor = blue
}

\bibliographystyle{unsrtnat}
\renewcommand{\familydefault}{\sfdefault}

\PassOptionsToPackage{hyphens}{url}

\renewenvironment{abstract}
 {{\bfseries\noindent{\abstractname}\par\nobreak}\footnotesize}
 {\bigskip}

\titlespacing{\section}{0pt}{*3}{*1}
\titlespacing{\subsection}{0pt}{*2}{*0.5}
\titlespacing{\subsubsection}{0pt}{*1.5}{0pt}

\providecommand\citet{\cite}
\providecommand\citep{\cite}
\providecommand\citealt{\cite}
\renewcommand\cite{\citep}

% You can conditionalize code for latexml or normal latex using this.
\newif\iflatexml\latexmlfalse
\providecommand{\tightlist}{\setlength{\itemsep}{0pt}\setlength{\parskip}{0pt}}%

\AtBeginDocument{\DeclareGraphicsExtensions{.pdf,.PDF,.eps,.EPS,.png,.PNG,.tif,.TIF,.jpg,.JPG,.jpeg,.JPEG}}

\newglossary[odg]{oalgo}{old}{odn}{OHD-GAN Acronyms}
\glsxtraddallcrossrefs
\makeglossaries

\begin{document}

    
%%%%%%%%%%%%%%%%%%%%%%%%%%% Algorithms
\newglossaryentry{gumbel-gan}
{
        name=Gumbel-Softmax GAN,
        description={}
}

\newglossaryentry{arae}
{
        name=ARAE,
        description={Adversarially regularized autoencoders.}
}

\newglossaryentry{t-sne}{
    name=t-SNE,
    description={The t-Distributed Stochastic Neighbor Embedding clustering algorithm is a nonlinear dimensionality reduction technique commonly applied to high-dimensiona data. See \citet{maaten2008tsne}.}
}

\newglossaryentry{mode-collapse}
{
        name=mode collapse,
        description={The training procedure fails to converge, or converges to an undesirable local minima resulting in a lack of variety in the generated samples.}
}

\newglossaryentry{feed-forward}
{
        name=feed-forward network,
        description={Basic Neural Network in its simplest form.}
}

\newglossaryentry{mb-avg}
{
        name=Mini-batch averaging,
        description={Adaptation of mini-batch averaging to cope with mode collapse, see \cite{choi2017generating}}
}

\newglossaryentry{dom-tran}{
    name=domain translation,
    description={Transforming data points from one domain or category to another.}
}

\newglossaryentry{semi-sup}{name=semi-supervised, description={\todo{definition}
}}

\newglossaryentry{re-iden}{name=reidentification attack, description={\todo{definition}
}}

\newglossaryentry{dbio}{name=digital bio-markers, description={\todo{definition}
}}

\newglossaryentry{exploding}{name=exploding gradient,
    description={The gradients accumulate large amounts of error, destabilising or disabling the training procedure.}
}

\newglossaryentry{vanishing}{name=vanishing gradient, description={The gradients become null and the network can no longer be updated.
}}

\newglossaryentry{a-disclosure}{name=Attribute disclosure, description={\todo{definition}}}

\newglossaryentry{mem-inference}{name=Membership inference, description={\todo{definition}}}

\newglossaryentry{repro-rate}{name=Reproduction rate, description={\todo{definition}}}

\newglossaryentry{utility-metric}{name=utility-based metric, description={In a broad sense, any metric that measure the amount of work that can be done with the data}}

\newglossaryentry{iteff}{name=Individual Treatment Effects, description={\todo{definition}
}}
\newglossaryentry{pmed}{name=Personalized Medicine, description={\todo{definition}
}}

\newglossaryentry{}{name=, description={
}}

%% Missingness
\newglossaryentry{mar}{name=Missing at Random, description={Given a dataset with missing entries , the missingness depends only on the observed variables \cite{yoon2018imputation}.
}}

\newglossaryentry{mcar}{name=Missing Completly at Random, description={Given a dataset with missing entries, the missingness is not depedant on any of the variables, thus occurs completly at random \cite{yoon2018imputation}.
}}

%% Self and co-training semi-supervised training
\newglossaryentry{co-training}{name=co-training, description={The self-training and co-training methods use classifiers first trained on the portion of labelled data to predict the labels of unlabelled instances. The newly labelled samples with the highest confidence are added to the labelled set to retrain the classifiers. The process is repeated iteratively.}}

\newglossaryentry{self-training}{name=self-training, description={The self-training and co-training methods use classifiers first trained on the portion of labelled data to predict the labels of unlabelled instances. The newly labelled samples with the highest confidence are added to the labelled set to retrain the classifiers. The process is repeated iteratively.
}}

%% Privacy
\newglossaryentry{mia}{name=Membership Inference Attack, description={Broadly, an MIA attack aims to determine if a particular record was used to train a machine learning model \cite{chen2019ganleaks}. There is no canonical process by which an attack is conducted, nor specification of the data assets initially in possession of the attacker. For a comprehensive taxonomy of MIA against \gls{gan}, refer to the suitably titled publication by \citeauthor{chen2019ganleaks} in which \gls{medgan} was subjected to a number of trials.}}

%% Training techniques
\newglossaryentry{msn}{
 type=\acronymtype,
 name={MSN},
 description={Per feature, a variational Gaussian mixture model is used to estimate the number of modes and fit a Gaussian mixture. A one-hot vector indicating the mode, and a scalar indicating the value within the mode is produced. See \cite{Xu2019-ay}.},
 text={MSN},
 first={Mode-specific normalization (MSN)}
}

\newglossaryentry{tbs}{
    type=\acronymtype,
    name={TbS}, 
    description={To deal with the imbalance of values in categorical featues, during training the data is resampled in a way that all the categories from discrete attributes are sampled evenly, without inducing bias and so as to recover real data distribution. See \cite{Xu2019-ay} for a step-by-step spefication.}
    text={TbS},
    first={Training by sampling (TbS)}  
}

\newglossaryentry{nnaa}{
    type=\acronymtype,
    name={NN-AA}, 
    description={"Compares the distance from one point in a target distribution T, to the nearest point in a source distribution S, to the distance to the next nearest point in the target distribution." See \cite{yale2019ESANN}.}
    text={NN-AA},
    first={Nearest-neighbor Adversarial Accuracy (NN-AA)}
}

\newglossaryentry{pl}{
    type=\acronymtype,
    name={PL}, 
    description={Difference of \gls{NN-AA} on the test set and on the training set. See \cite{yale2019ESANN}.}
    text={PL},
    first={Privacy loss (PL)}
}

\newglossaryentry{dt}{
    type=\acronymtype,
    name={DT}, 
    description={The discriminator is tested on batches of synthetic data produced by other methods to asses the possibility of overfitting, see \cite{yale2019ESANN}.}
    text={DT},
    first={Discriminator testing (DT)}
}

\newglossaryentry{do}{
    type=\acronymtype,
    name={DO}, 
    description={Privacy preservation method. See \cite{yale2019ESANN} based on \cite{Dwork2008, Prasser2017}.}
    text={DO},
    first={Data obfuscation (DO}
}

\newglossaryentry{pate}{
    type=\acronymtype,
    name={PATE}, 
    description={Differntial privacy method: "The approach combines, in a black-box fashion, multiple models trained with disjoint datasets, such as records from different subsets of users. Because they rely directly on sensitive data, these models are not published, but instead used as "teachers" for a "student" model. The student learns to predict an output chosen by noisy voting among all of the teachers, and cannot directly access an individual teacher or the underlying data or parameters. The student's privacy properties can be understood both intuitively (since no single teacher and thus no single dataset dictates the student's training) and formally, in terms of differential privacy." \cite{Papernot2017,Papernot2018}}
    text={PATE},
    first={Private Aggregation of Teacher Ensembles (PATE)}
}

\newglossaryentry{mbd}{
    type=\acronymtype,
    name={MBD}, 
    description={Training technique. See \cite{Salimans2016}}
    text={MBD},
    first={Mini-batch discrimination (MDB)}
}

\newglossaryentry{t-gan}{
    type=\acronymtype,
    name={T-GAN}, 
    description={Training technique to stabilise training. Allows the introduction of real sample information into the process of training the the generator. See \cite{Jolicoeur-Martineau2019, Su2018}}
    text={T-GAN},
    first={Turing \gls{gan}}
}

\newglossaryentry{corrnn}{
    type=\acronymtype,
    name={CorrNN}, 
    description={Learns a common representation of two views, taking into account their correlation. See \cite{Jolicoeur-Martineau2019, Su2018}}
    text={CorrNN},
    first={Correlation \gls{nn}}
}

\newacronym{}{Grouped CorrNN}{Grouped Correlation Neural Network}

% \newglossaryentry{⟨label ⟩}{type=\acronymtype,
% name={⟨abbrv ⟩},
% description={⟨long⟩},
% text={⟨abbrv ⟩},
% first={⟨long⟩ (⟨abbrv ⟩)},
% plural={⟨abbrv ⟩\glspluralsuffix},
% firstplural={⟨long⟩\glspluralsuffix\space (⟨abbrv ⟩\glspluralsuffix)},
% ⟨key-val list⟩}

%% Algorithms
\newacronym{nn}{NN}{Neural Network}
\newacronym{gan}{GAN}{Generative Adversarial Network}
\newacronym{ohd-gan}{OHD-GAN}{\glspl{gan} for Observation Health Data}
\newacronym{ffn}{FFN}{\gls{feed-forward} Network}
\newacronym{ae}{AE}{Autoencoder}
\newacronym{rnn}{RNN}{Reccurent \gls{nn}}
\newacronym{lstm}{LSTM}{Long Short-term Memory}
\newacronym{cgan}{CGAN}{Conditional \gls{gan}}
\newacronym{crmb}{CRMB}{Conditional Restricted Boltzamann Machine}
\newacronym{cnn}{CNN}{Convolutional \gls{nn}}
\newacronym{wgan}{WGAN}{Wassertein \gls{gan}}
\newacronym{beta-vae}{\ensuremath{\beta}-VAE}{\ensuremath{\beta} variational auto-encoder}
\newacronym{lr}{LR}{Logistic-regression}
\newacronym{cycle-gan}{Cycle-GAN}{Cycle-consistent \gls{gan}}
\newacronym{adtep}{ADTEP}{Adversarial Deep Treatment Effect Prediction}
\newacronym{cae}{CAE}{Convolutional \gls{AE}}

\newacronym[type=oalgo]{medgan}{medGAN}{medGAN}
\newacronym[type=oalgo]{ssl-gan}{SSL-GAN}{Semi-supervised Learning with a learned ehrGAN}
\newacronym[type=oalgo]{wgantpp}{WGANTPP}{\gls{wgan} for Temporal Point-processes}
\newacronym[type=oalgo]{radialgan}{RadialGAN}{RadialGAN}
\newacronym[type=oalgo]{mc-arae}{MC-ARAE}{Multi-categorical gls{arae}}
\newacronym[type=oalgo]{ctgan}{CTGAN}{Conditional Tabular \Gls{gan}}
\newacronym[type=oalgo]{heterogan}{HGAN}{Heterogeneous GAN}
\newacronym[type=oalgo]{emr-wgan}{EMR-WGAN}{EMR Wassertein GAN}
\newacronym[type=oalgo]{corgan}{corGAN}{corGAN}
\newacronym[type=oalgo]{1d-cae}{1D-CAE}{1-dimensional Convolutional \gls{ae}}
\newacronym[type=oalgo]{ehrgan}{ehrGAN}{Electronic Health Record GAN}
\newacronym[type=oalgo]{rgan}{RGAN}{Recurrent \gls{gan}}
\newacronym[type=oalgo]{rcgan}{RC-GAN}{Recurrent Convolutional \gls{gan}}
\newacronym[type=oalgo]{ganite}{GANITE}{Generative Adversarial Nets for inference of Individualized Treatment Effects}
\newacronym[type=oalgo]{cwr-gan}{CWR-GAN}{Cycle Wasserstein Regression \gls{gan}}
\newacronym[type=oalgo]{gain}{GAIN}{Generative Adversarial Imputation Network}
\newacronym[type=oalgo]{mc-medgan}{MC-medGAN}{Multi-categorical \gls{medgan}}
\newacronym[type=oalgo]{mc-gumbelgan}{MC-GumbelGAN}{Multi-categorical Gumbel-softmax \gls{gan}}
\newacronym[type=oalgo]{mc-wgan-gp}{MC-WGAN-GP}{Multi-categorical \gls{wgan} with Gradient Penality}
\newacronym[type=oalgo]{medbgan}{MedBGAN}{Boundry-seeking \gls{medgan}}
\newacronym[type=oalgo]{healthgan}{HealthGAN}{}
\newacronym[type=oalgo]{medwgan}{MedWGAN}{Wassertein \gls{medgan}}
\newacronym[type=oalgo]{sc-gan}{SC-GAN}{Sequentially Coupled \gls{gan}}
\newacronym[type=oalgo]{rmb}{RMB}{Restricted Boltzmann Machine}
\newacronym[type=oalgo]{anomigan}{AnomiGAN}{GANs for anonymizing private medical data}
\newacronym[type=oalgo]{wgan-gp}{WGAN-GP}{\gls{wgan} with Gradient Penality}
\newacronym[type=oalgo]{dp-auto-gan}{DP-auto-GAN}{\gls{DP}-auto-\gls{gan}}
\newacronym[type=oalgo]{ads-gan}{ADS-GAN}{Anonymization through data synthesis using \gls{gan}}
\newacronym[type=oalgo]{gcgan}{GcGAN}{\gls{corrnn} and \gls{t-wgan}}
\newacronym[type=oalgo]{t-wgan}{T-wGAN}{Wassertein \gls{t-gan}}
\newacronym[type=oalgo]{conan}{CONAN}{\textit{Co}plementary patter\textbf{n A}augmentatio\textbf{n } }

%\newacronym[type=oalgo]{}{}{}
%\newacronym{}{}{}

%% Terms
\newacronym{sd}{SD}{Synthetic Data}
\newacronym{ohd}{OHD}{Observational Health Data}
\newacronym{ehr}{EHR}{Electronic Health Record}
\newacronym{icu}{ICU}{Intensive Care Unit}
\newacronym{pmf}{PMF}{Probability Mass Function}


%% Fields
\newacronym[seealso=iteff]{ite}{ITE}{Individual Treatment Effects}
\newacronym[seealso=iteff]{dle}{DLE}{Drug Laboratory Effects}
\newacronym[seealso=pmed]{pm}{PM}{Personalized Medicine}
\newacronym{iot}{IoT}{Internet of Things}

%% Techniques
\newacronym{mba}{MbA}{\gls{mb-avg}}
\newacronym{bn}{BN}{batch-normalization}
\newacronym{sc}{SC}{shortcut connections}
\newacronym{cbt}{CBT}{Cluster-based training}
\newacronym{vcd}{VCD}{Variational contrastive divergence}
\newacronym{ln}{LN}{Layer normalisation}
\newacronym{ssl}{SSL}{Semi-supervised Learning}
\newacronym{sn}{SN}{Spectral Normalization}


%% Privacy
\newacronym{dp}{DP}{Differential privacy}
\newacronym{dp-sgd}{DP-SGD}{Differential private stochastic gradient descent}
\newacronym{ad}{AD}{Attribute Disclosure}
\newacronym[seealso=mia]{pd}{PD}{Presence Disclosure}
\newacronym{rr}{RR}{Reproduction rate}
\newacronym[seealso=mia]{mi}{MI}{Membership Inference}

\newacronym{anm}{ANM}{Additive noise model}


%% Evaluation qualitative
\newacronym{ved}{VED}{Visual Expert Discrimination}

%% Evaluation statistics
\newacronym{dwpro}{DWS}{Dimension-wise Statistics}
\newacronym{dwpre}{DWP}{Dimension-wise Prediction}

%% Evaluation quantitative
\newacronym{fd}{FD}{Feature distributions}
\newacronym{qq}{QQ}{Quantile-quantile plot}
\newacronym{lsr}{LSR}{Latent space representation}
\newacronym{rdp}{RDP}{Renyi Differential Privacy}
\newacronym{pcam}{PCAM}{\gls{pca} Marginal}
\newacronym{pca}{PCA}{Principal Component Analysis}
\newacronym{pcawdd}{PCA-DWD}{\gls{pca} Distributinal Wassertein Distance}

%% Metrics
\newacronym{fop}{F-OP}{First-order proximity}
\newacronym{cc}{CC}{Correlation coefficient}
\newacronym{md-cc}{MD-CC}{\todo{Redundant?}}
\newacronym{mmd}{MMD}{Maximum Mean Discrepency}
\newacronym{rbf}{RBF}{Radial Basis Function}
\newacronym{mse}{MSE}{Mean Squared Error}
\newacronym{auroc}{AUROC}{Area under ROC curve}
\newacronym{auprc}{AUPRC}{Area under the precision-recall curve}
\newacronym{kld}{KLD}{Kullback-Leibler divergence}

%% Evaluation augmentation
\newacronym{tstr}{TSTR}{Train on synthetic, test on real}
\newacronym{trts}{TRTTS}{Train on real, test on synthetic}
\newacronym{pta}{PTA}{Prediction task accuracy}
\newacronym{ssa}{SSA}{Semi-supervised augmentation}






    \title{Generative Adversarial Networks Applied to Observational Health Data}
    
    \author[1,2]{Jeremy Georges-Filteau}%
    \author[2]{Elisa Cirillo}%
    \affil[1]{Radboud University Nijmegen}%
    \affil[2]{The Hyve}%


    \vspace{-1em}

    \date{\today}

    \begingroup
    \let\center\flushleft
    \let\endcenter\endflushleft
    \maketitle
    \endgroup

    \selectlanguage{english}


    \begin{abstract}
    Having been collected for its primary purpose in patient care, \gls{ohd} can further benefit patient well-being
    by sustaining the development of health informatics.~ However, the potential for secondary usage of OHD continues to be hampered by the fiercely private nature of patient-related data. ~ gls{gan} have recently emerged as a groundbreaking approach to
    efficiently learn generative models that produce realistic \gls{sd}. However, the application of \gls{gan} to \gls{ohd} seems to have been lagging in comparison to other fields. We conducted a review of \gls{gan} algorithms for OHD in the published literature, and report our findings here.
    \end{abstract}

    \section{Introduction}
\subsection{Background}
Most \gls{ohd} is collected as \glspl{ehr} at various points of care in a patient’s trajectory, primarily to support and enable healthcare professionals \cite{Cowie_2016}. The patient profiles found in \glspl{ehr} are diverse and longitudinal, composed of demographics variables, recordings of diagnoses, conditions, procedures, prescriptions, measurements and lab test results, administrative information, and increasingly omics \cite{Ohdsi2020-vf}.\par
Having served its primary purpose, this wealth of detailed information can further benefit patient well-being by sustaining medical research and development. This could mean improving the development life-cycle of health informatics (HI), the predictive accuracy of machine learning (ML) algorithms or enabling discoveries in research concerning clinical decisions, triage decisions, inter-institution collaborations and HI automation \cite{Rudin_2020}. Big health data is the underpinning of two main objectives of precision medicine: individualization of patient interventions and the inference of biological systems from high level analysis \cite{Capobianco2020}. However, the potential for secondary usage of \gls{ohd} continues to be hampered by the fiercely private nature of patient-related data, and the growing popular concern towards its disclosure.\par
Anonymization techniques are generally employed to hinder misuse of sensitive data. Most often, through a costly and data specific cleansing process, privacy is enhanced at the detriment of data utility. Moreover, these techniques are fallible, and never fully prevent re-identification.To address this problem, alternative methods for sharing sensitive data have been proposed, such as privacy-preserving distributed analysis. Although promising, these approaches come with their own limitations.\par
Consequently, access to \gls{ohd} is restricted to professionals with the appropriate academic credentials and financial resources, preventing its use for the rest of the health data related occupations. For example, software developers often do not have access to the data that will be processed by the health informatics solutions they are developing.
\subsection{Synthetic data}
An alternative to traditional privacy-preserving methods is to produce fully synthetic data,with methods to build these models including knowledge-driven and data-driven modelling \cite{Kim_2017}. Knowledge-driven modelling involves a complex theory-based process to define a simulation process representing the causal relationships of a system. The Synthea \cite{Walonoski_2017} synthetic patient generator is one such simulation model, in which predefined states, transitions, and conditional logic produce patient trajectories. The parameters of the Synthea model are taken from aggregate population-level statistics of disease progression and medical knowledge. A knowledge-based approach such as Synthea depends on prior knowledge of the system, and most importantly how much we can understand about it \cite{Kim_2017}. When modelling complex systems, simplifications and assumptions are inevitable, leading to inaccuracies. For example, relying on population-level statistics does not produce models capable of reproducing heterogeneous health outcomes \cite{Chen_2019}.\par
In data-driven modelling techniques, a representation of the data is inferred from a sample distribution. There exists numerous statistical modelling approaches to produce synthetic data, but the modelling processes are based on intrinsic assumptions about the data, the representational power is bound to the correlations that are intelligible to the modeler or are prone to obscure inaccuracies. Synthetic data generated by these models tends to possess low utility \cite{Rankin2020}. In the ML field, generative models learn to represent an estimate of the multi-modal distribution, from which synthetic samples can be drawn \cite{goodfellow2016nips}. Generative Adversarial Networks (GAN) \cite{NIPS2014_5423} have recently emerged as a groundbreaking approach to efficiently learn generative models that produce realistic Synthetic Data (SD) using \gls{nn}. GAN algorithms have rapidly found a wide range of applications, such as data augmentation in medical imaging \cite{Kadurin_2017}.\par
The potential impacts of GAN to healthcare and science are considerable, some of which have been realized in fields such as medical imaging \cite{Yi_2019}. However, the application of GAN to \gls{ohd} seems to have been lagging \cite{Xiao_2018}. Certain characteristics of \gls{ohd} could serve to explain the relatively slow progress. Primarily, algorithms developed for images and text in other fields were easily re-purposed for medical equivalents. However, \gls{ohd} presents unique complexity in terms of multi-modality, heterogeneity and fragmentation \cite{Xiao_2018}. In addition to this, evaluating the realism of synthetic \gls{ohd} is intuitively complex, a problem that still burdens GAN in general. Nonetheless, interesting GAN solutions to the challenges posed by \gls{ohd} have been developed \cite{esteban2017real,Che_2017,choi2017generating,yahi2017generative}.
    \section{Methods}

    \begin{table}[h]
  \center
  \footnotesize
    \caption{Search query terms}\label{tab:search}
    \begin{tabular}{@{}clccl@{}} \toprule
	    \multicolumn{2}{c}{Health data} & & \multicolumn{2}{c}{Generative adversarial models} \\ \cmidrule{1-2} \cmidrule{4-5}
	    \multicolumn{2}{c}{Terms} & {} & \multicolumn{2}{c}{Terms} \\ \cmidrule{2-2} \cmidrule{5-5}
	    \multirow{4}{*}{OR} & clinical & \multirow[t]{4}{*}{\quad AND\quad} & \multirow{4}{*}{OR} & generative adversarial\\
	    {} & health & {} & {} & GAN \\ 
	    {} & EHR & {} & {} & adversarial training \\
	    {} & electronic health record & {} & {} & synthetic  \\
	    {} & patient & {} & {} & {} \\
	    \bottomrule
    \end{tabular}
\end{table}
    
    Publications concerning \gls{ohd-gan} were identified through with Google Scholar \cite{scholar}, Web of Science \cite{Clarivate} and Prophy \cite{Prophy}. The terms and operators found in Table \ref{tab:themes} form the search query. We included studies reporting the development, application, performance evaluation and privacy evaluation of \gls{gan} algorithms to produce \gls{ohd}. We define \gls{ohd} as categorical, real-valued, ordinal or binary event data recorded for patient care. We list a more detailed summary of the included and excluded data types in Table \ref{tab:datatypes}. The excluded data types are already the subject of one or more review, or would merit a review of their own \cite{Yi_2019, Nakata2019, Anwar_2018, Wang2020, Zhou2020}. In each of the included publications, we considered the aspects listed in Table \ref{tab:search}.\par

        \input{tables/themes}
    
    \begin{table}[htp]
\center
\footnotesize
  \caption{Types of OHD data included or excluded from the review.}\label{tab:datatypes}
  
  \begin{tabularx}{\textwidth}{@{} p{0.1\textwidth}Xp{0.7\textwidth}@{}}\toprule
  Type & Examples \\ \midrule
  
  \multirow{4}{*}{Included} & Observations & Demographic information, medical classification, family history \\
  
  &Timestamped observations & Diagnosis, treatment and procedure codes, prescription and dosage, laboratory test results, physiologic measurements and intake events \\
  &Encounters & Visit dates, care provider, care site \\
  &Derived & Aggregated counts, calculated indicators. \\ \midrule

  \multirow{4}{*}{Excluded} &Omics & Genome, transcriptome, proteome, immunome, metabolome, microbiome \\
  &Imaging & X-rays, computed tomography (CT), magnetic resonance imaging (MRI) \\
  &Signal & Electrocardiogram (ECG), electroencephalogram (EEG) \\
  &Unstructured & Narrative reports, textual \\ \bottomrule
  \end{tabularx}%
\end{table}
  
    
    








    \section{Results}
    \subsection{Summary}
        We have found a total of \todo{count the number of publications} publications describing the development or adaption of \gls{gan} algorithms for \gls{ohd}, presented in Table \ref{tab:publications}. The type of data addressed in each of these publications can be generalized into one of two categories: time-dependent observations, such as time-series, or static representation in the form of feature vectors. Publications considering privacy either perform privacy evaluations of their algorithms and synthetic data, or exclusively concentrate on comparing methods about privacy.
        
        The most efforts are focused on adapting the current methods to the characteristics and complexities of OHD, of which multi-modality or non-Gaussian continuous features, heterogeneity, a combination of discrete and continuous features, longitudinal irregularity, correlation complexity, missingness or sparsity, class imbalance and noise are often cited. While these properties may pose a challenge for the development of useful algorithms, others aspects make the prospect of success highly valuable. In fact, the most cited motive to develop \gls{ohd-gan} is to cope with the often limited number of samples in medical datasets and to overcome the highly restricted access to \gls{ohd}.\par
        
        \subsection{Motives for developing \gls{ohd}-\gls{gan}}
The authors mention a wide range of potential applications for \gls{ohd-gan}. While some of these goals are overoptimistic and have yet to be realized, they paint an encouraging picture for the value of synthetic \gls{ohd} and the transformative effect it could have on healthcare initiatives and scientific progress. We briefly describe the four prevailing themes in the following sections: data augmentation (Sec.\ref{sec:augmentation}), privacy and accessibility (Sec.\ref{sec:access_privacy}), precision medicine (Sec.\ref{sec:precision_med}) and  modelling simulations (Sec.\ref{sec:models_twins}). 

    \subsubsection{Data augmentation}\label{sec:augmentation}
    
    Data augmentation is mentioned in nearly all publications. Although counter-intuitive, it is well known that \glspl{gan} can generate \gls{sd} that conveys more information about the real data distribution. Effectively, the continuous space distribution of the generator produces a more comprehensive set of data points, valid but not present in the discrete real data points. A combination of real and synthetic training data habitually leads to increased predictor performance \cite{Wang_2019,Che_2017,Yoon2018-ite, yoon2018imputation}. A more intelligible way to seize the concept from the point of view of image classification, in which it is known as invariances, perturbations such as rotation, shift, sheer and scale \cite{antoniou2017data}. Similarly, domain translation and \gls{semi-sup} training approaches could support predictive tasks that lack data with accurate labels, paired samples, or suffer from class imbalance \cite{Che_2017,mcdermott2018semi}. 

    \subsubsection{Enhancing privacy and increasing data accessibility}\label{sec:access_privacy}
    
    \gls{sd} is seen as the key to unlocking the unexploited value of \gls{ohd} due to privacy concerns. Preserving privacy can broadly be described as reducing the risk of \glspl{re-iden} to an acceptable level. This level of risk is quantified when releasing data anonymized with \gls{dp}. Authors noted that highly restricted access to \gls{ohd} is hindering machine learning, and more generally scientific progress \cite{Beaulieu-Jones2019-ct, Baowaly_2019,Che_2017,esteban2017real,Fisher2019}.\par
    
    Due to its artificial nature, SD is put forward as a means to forgo data use agreements, while potentially providing greater privacy guarantees\cite{Beaulieu-Jones2019-ct, baowaly_2019_IEEE, baowaly_2019_jamia,esteban2017real,Fisher2019,walsh2020generating}. In fact, \gls{gan} training according to \gls{dp} shows evidence of reducing the loss of utility in comparison to \gls{dp} alone. \todo{Find these citations} Overall, enabling access to greater variety, quality and quantity of \gls{ohd} could have positive effects in a wide range of fields, such as software development, education, and training of medical professionals. 
    
    \subsubsection{Enabling precision medicine}\label{sec:precision_med}
    
    The ability to conduct personalized simulations of disease progression for individual patients could have transformative impacts on healthcare. Generative models able to produce time-series trajectories conditioned on a patient's baseline state could help inform clinical decision making by quantifying disease progression and outcomes \cite{walsh2020generating, Fisher2019}. Ensembles of stochastic simulations of individual patient profiles such as those produced by gls{crmb} could help quantify risk at an unprecedented level of granularity \cite{Fisher2019}.\par
    Predicting patient-specific responses to drugs is still a new field of research, a problem known as \gls{ite}. The task of estimating \glspl{ite} is persistently hampered by the lack of paired counterfactual samples \cite{Yoon2018-ite, chu2019treatment}. In medical imaging, various \gls{gan} algorithms were developed for domain translation, mapping a sample from its to original class to the paired equivalent. This includes bidirectional transformations, allowing \glspl{gan} to learn mappings from very few, or a lack of paired samples \cite{Wolterink2017DeepMT}.
    
    \subsubsection{From patient and disease models to digital twins}\label{sec:models_twins}
    
    A well trained model approximates the process that generated the real data points \cite{esteban2017real}. In other words, the relations learned by the model, its parameters, contains meaningful information if we can learn to harness it. Interpretability is a growing field of research concerned with understanding how these learned parameters relate, and thus explaining the representations the algorithm has converged to in linking the features to the outcome.\par
    Achieving models of significant complexity would both open up unprecedented simulation capabilities, but also the chance to explore meaningful representations that would otherwise be beyond our reasoning.\par 
    In clinical research, such models could help quantify cause and effect, simulate different study designs, provide control samples or more generally give us a better understanding of disease progression in relation to initial conditions \cite{Fisher2019, yahi2017generative, walsh2020generating}.\par
    
    Approaching these ideas from above, the concept of "digital twins" represents in a way the ultimate realization of \gls{pm}. A common practice in industrial sectors is high-fidelity virtual representations of physical assets. Long-term simulations, that provide an overview and comprehensive understanding of the workings, behavior and life-cycle of their real counterparts. The state of the models is continuously updated from theoretical data, real data and streaming \gls{iot} indicators.\par
    Intently conditioned input data allows the exploration of specific events or conditions. In a position paper on the subject, Angulo et al. draw the parallels of this technique with the current needs in healthcare and the emergence of the necessary technologies for actionable models of patients. \cite{angulo2019towards,Angulo_2020}. The authors bring up the rapid adoption of wearables that are continuously monitoring people's physiological state. 
    Wearables are one of many mobile digitally connected devices that collect patient data over a broad range of physiological characteristic and behavioral patterns \cite{coravos2019developing}. This emerging trend known as \gls{dbio} has already led to studies demonstrating predictive models with the potential for improved patient care \cite{snyder2018best}. Through continuous lifelong learning, integrating  multiple modes of personal data, generative patient models could inform diagnostics of medical professionals and also enable testing treatment options. In their proposal, \glspl{gan} are an essential component of the ecosystem to ensure patient privacy and to provide bootstrap data. Notably, Fisher et al. already employ the term "digital twin" to describe their process \cite{walsh2020generating}.
        
\begin{sidewaystable}[htpb]
\scriptsize
  \centering
    \caption{Publications included}\label{tab:publications}
  
    \begin{tabularx}{\textwidth}{@{}p{3cm}XXXXXX@{}} \toprule
    Publication 
    & Algorithm 
    & Focus 
    & Data type 
    & Algorithm and Techniques 
    & Evaluation 
    & Privacy\\ \midrule
    
    \cite{Choi2017-nt}
    & \thealgo{medGAN} 
    & Discrete features, Mode collapse 
    & Binary, Ordinal 
    & \gls{ffn},\gls{ae}, \gls{mba}, \gls{bn}, \gls{sc} 
    & \gls{dwpre}, \gls{dwpro} 
    & \gls{ad}, \gls{pd}\\
    
    \cite{esteban2017real}
    & \thealgo{RGAN}, \thealgo{RCGAN} 
    & Real-valued time-series, conditional training 
    & Regulary obsverved time-series 
    & \gls{lstm}, \gls{swl}, \gls{cgan} 
    & \gls{mmd}, \gls{tstr}, \gls{trts}, \gls{auroc}, \gls{auprc}, \gls{pta} 
    & \gls{dp-sgd}\\
    
    \cite{yahi2017generative} 
    & $-$
    & Continuous time-series, Drug laboratory effects (DRE) 
    & Paired pre and post exposure time-series 
    & \algo{medGAN}, Clustering, t-SNE 
    & \gls{mse}
    & $-$ \\
    
    \cite{Che_2017} 
    & \thealgo{ehrGAN}, \thealgo{SSL-GAN} 
    & Discrete time-series, semi-supervised augmentation 
    & Sequences of medical codes 
    & 1D-CNN, Word2vec, \gls{vcd}  
    &\gls{cc}, \gls{fd}, \gls{ssa}
    & $-$\\
    
    \cite{Xiao2017-lh} 
    & \thealgo{WGANTPP} 
    & Temporal Point Processes 
    & Sporadic occurences 
    & \gls{lstm}, \gls{wgan} 
    & Poisson process 
    & \gls{qq} \\
    
    \cite{Yoon2018-dm}
    & \thealgo{RadialGAN} 
    & Multi-domain translation, features and distribution mismatch, cycle-consistency, augmentation 
    & Tabular, discrete and continuous 
    & \gls{ffn} 
    & \gls{cgan}, \gls{wgan}, \gls{md-cc} 
    & \gls{pta}, \gls{auroc}, \gls{auprc} \\
    
    \cite{Yoon2018-mo} 
    & \thealgo{GANITE} 
    & \gls{ite}, unobserved counterfactuals 
    & Feature, treatment and outcome vectors 
    & \gls{cgan} pair 
    & See publication 
    & See publication \\
    
    \cite{Camino2018-re} 
    &\thealgo{MC-ARAE}, \thealgo{MC-medGAN}, \thealgo{MC-GumbelGAN}, \thealgo{MC-WGAN-GP} 
    & Data composed of multiple categorical variables 
    & Multiple categorical variables represented as one-hot encoded vectors 
    & \algo{medGAN}, \algo{WGAN-GP}, \gls{gumbel-gan}, \gls{arae} 
    & See Section \ref{sec:categorical} 
    & $-$\\
    
    \cite{Zhang2020}
    &\thealgo{EMR-WGAN}
    & Medical codes, improving training, evaluation metrics
    & Binary vector of occurrence over the medical codes. Low-prevalence of codes by which approximately half the dataset is discarded.
    & \gls{wgan}, \gls{bn}, \gls{ln}, \gls{cgan}
    & \gls{dwpro}, \gls{dwpre}, \gls{lsr}, \gls{fop}
    &\gls{ad}, \gls{mi}, \gls{rr}\\
    
    \bottomrule    \end{tabularx}
\end{sidewaystable}
        
        \subsection{Data oriented GAN development}
\subsubsection{Auto-encoders and categorical features}
In what is, to the best of our knowledge the first attempt at developing a GAN for OHD. Choi et al. focus on the problem posed by the incompatibility of categorical and ordinal features with back-propagation. Their solution is to pretrain an Autoencoder (AE) to project the samples to and from a continuous latent space representation and retain the decoder portion to form a component of the GAN \cite{Choi2017-nt}. In the algorithm \thealgo{medGAN,} the trained decoder in incorporated into the generator and maps the randomly sampled input vectors from latent space representation back to discrete features. This first exemplar of synthetic OHD generated by GAN inspires a series of enhancements.\par

Numerous efforts were made to improve the performance of \algo{medGAN}. Among the first, Camino et al. developed \thealgo{MC-medGAN} in which they modified the AE by adding a Gumbel-Softmax \cite{jang2016categorical} activation layer after splitting its output with a dense layer for each categorical variable and finally concatenating of the Gumbel-Softmax \todo layers \cite{Camino2018-re}. The authors also developed an adaptation based on recent training techniques: Wassertein GAN (WGAN) \cite{arjovsky2017wasserstein} and a WGAN with Gradient Penalty (WGAN-GP) \cite{gulrajani2017improved}. In brief, the Wasserstein distance is a measure of distance between two probability distributions that has the property of always providing a smooth gradient. When used as the loss function of the discriminator, it generally improves training stability and mitigates mode collapse. Weight clipping is used in WGAN to ensure the discriminator lies within of 1-Lipschitz functions. The undesirable effects of weight clipping are eliminated by rather imposing a penalty on the gradient . \thealgo{MC-WGAN-GP} is the equivalent of \algo{MC-medGAN} but with Softmax layers. The authors report that the choice of a model will depend on data characteristics, particularly sparsity.\par 
Wasserstein's distance was widely adopted by subsequent authors for its compatibility with OHD. Baowaly et al. developed \thealgo{MedWGAN} also based on WGAN, and \thealgo{MedBGAN} borrowing from Boundary-seeking GAN (BGAN) \cite{hjelm2017boundaryseeking} which pushes the generator to produce samples that lie on the decision boundary of the discriminator, expanding the search space. Both led to improved data quality, in particular \algo{MedBGAN} \cite{baowaly_2019_IEEE,baowaly_2019_jamia}. In other effort, Jackson et al. tested \algo{medGAN} on an extended dataset containing demographic and health system usage information, obtaining results similar to those of the original \cite{Jackson_2019}. The \thealgo{HealthGAN} built upon WGAN-GP, but includes a data transformation method adapted from the Synthetic Data Vault \cite{Patki_2016} to map categorical features to and from the unit numerical range \cite{Yale_2020}. 
        
        \subsubsection{Forgoing the autoencoder}\label{noauto}

Suggesting the use of an \gls{ae} introduces noise, with \gls{emr-wgan}, Zhang et al. dispose of the \gls{ae} component of previous algorithms and introduce a conditional training method, along with conditioned \gls{bn} and \gls{ln} techniques to stabilise training \cite{Zhang2020-wp}. The algorithm was further adapted by Yan et al. as \gls{heterogan} to better account for the conditional distributions between multiple data types and enforce record-wise consistency. A recognized problem with \algo{medgan} was that it produced common-sense inconsistencies, such as gender mismatches in medical codes \cite{yan2020generating, choi2017generating}. In \gls{heterogan}, constraints are enforced by adding specific penalties to the loss function, such as limit ranges for numerical categorical pairs and mutual exclusivity for pairs of binary features \cite{yan2020generating}. \par

To develop \gls{ctgan}, Xu et al. presume that tabular data poses a challenge to GANs owing to the non-Gaussian multi-modal distribution of continuous columns and imbalanced discrete columns \cite{Xu2019-ay}. Their algorithm, composed of fully connected layers, was developed with adaptations to deal with both continuous and categorical features. For continuous features, it employs mode-specific normalization to capture the multiplicity of modes. For discrete features conditional training-by sampling is devised to resample discrete attributes evenly during training, while recovering the real distribution when generating data.\par

Other approaches include: \gls{corgan}, where the \gls{ae} is questionably replaced by a \gls{1d-cae} to capture neighboring feature correlations of the input vectors \cite{torfi2019generating}, and two basic feedforward networks based on Wassertein distance to evaluate the capacity of \glspl{gan} to model heterogeneous data of dense and sparse medical features \cite{chincheong2020generation} and to reproduce statistical properties \cite{ozyigit2020generation}. Reproducing physiological time-series \citeauthor{esteban2017real} used devise the \gls{rgan} and \gls{rcgan} based on \gls{lstm} to generate a regular time-series of physiological measurements from bedside monitors \cite{esteban2017real}. Curiously, the authors dismiss Wassertein's distance, stating that they did not find application in their experiments. In addition, each dimension of their time-series is generated independently from the others, where one would assume they are correlated. A considerable loss of accuracy is observed on their \gls{utility-metric}.

\subsection{Task oriented GAN development}
\subsubsection{Semi-supervised learning and conditional models}

To develop \gls{ehrgan}, an algorithm for sequences of medical codes that has the ability to produce neighbouring records of an input patient, \citeauthor{Che_2017} combine an Encoder-Decoder \gls{cnn} \cite{ranzato2007unsupervised} with \gls{vcd} \cite{Che_2017}. The \gls{ehrgan} generator is trained to decode a random vector mixed with the latent space representation of a particular patient. In a semi-supervised learning approach, the trained \gls{ehrgan} model is then incorporated into the loss function of a predictor where it can help generalization by producing neighbors for each input sample. Semi-supervised learning approaches are commonly employed to augment the minority class in imbalanced datasets, such as \gls{self-training} and \gls{co-training}.  Yang et al. improve on this type of approach by incorporating a GAN in the procedure \cite{yang}. The GAN is first trained on the labelled set and used to rebalance it. The standard iterative process involving the classifier ensemble is then executed until expansion ceases. As a final step, the GAN is trained on the expanded labelled set to generate an equal amount of augmentation data. The authors obtained improved performance in a number of classification tasks and multiple tablular datasets with their method.Correcting bias with domain translationTo address the heteogeneity of healthcare data from different sources, Yoon et al. combines the concepts of cycle-consistent domain translation from Cycle-GAN (Zhu 2017) and multi-domain translation from Star-GAN (Choi 2017a) to build RadialGAN to translate heterogeneous patient information from different hospitals, correcting features and distribution mismatches (Yoon 2018). An encoder-decoder pair per data endpoint is trained to map records to and from a shared latent representation. Individualized treatment effectsThe task of estimating Individualized Treatment Effects (ITE), the response of a patient to a certain treatment given a set of charaterizing features is an ongoing problem. This is due mainly to the fact that counterfactual outcomes are never observed or that treatment selection is highly biased (Yoon 2018a, McDermott 2018, Walsh 2020). In this regard, Yoon et al. employ a pair of GANs, named Generative Adversarial Nets for inference of Individualized Treatment Effects (GANITE), one for counterfactual imputation and another for ITE estimation (Yoon 2018a). The former captures the uncertainty in unobserved outcomes by generating a variety of conterfactuals. The output is fed to the latter, which estimates treatment effects and provides confidence intervals. With Cycle Wasserstein Regression GAN (CWR-GAN), a joint regression-adversarial model, McDermott et al. demonstrated a semi-supervised approach also inspired by Cycle-GAN to leverage large amounts of unpaired pre/post-treatment time-series in ICU data for the estimation of ITE on physiological time-series (McDermott 2018). The algorithm has the ability to learn from unpaired samples, with very few paired samples, to reversibly translate the pre and post-treatment physiological series. Chu et al. approach the problem of data scarcity for ITEs by designing ADTEP, an algorithm that can maximize use of the large volume of EHR data formed by triples of non-task specific patient features, treatment interventions and treatment outcomes (Chu 2019). The ADTEP algorithm they developed learns representation and discriminatory features of the patient, and treatment data by training an \gls{ae} for each pair of features. In addition to \gls{ae} reconstruction loss, a second model is tasked with identifying fake treatment feature reconstructions. Finally, a fourth loss metric is calculated by feeding the concatenated latent representations of both \gls{ae} to a logisitic regression model aimed at predicting the treatment outcome (Chu 2019). In the form of an ITE task, Wang et al. demonstrated an interesting algorithm to generate a time series of patient states and medication dosages using \gls{lstm}. In contrast to RGAN and RCGAN, in Sequentially Coupled Generative Adversarial Network (SC-GAN), patients state at the current timestep informs the concurrent medication dosage, which in turn affects the patient state in the upcoming timestep (Wang 2019). SC-GAN overcame a number of baselines on both statistical and utility metrics. Data Imputation with GANsGANs are naturally suited for data imputation, and could provide a new approach to deal with the problems of health data relating to sparsity. Statistical models developed for the multiple imputation problem increase quadraticly in complexity with the number of features, while the expressiveness of deep neural networks can model all features with missing values simultaneously efficiently. In that regard, Yoon et al. adapted the standard GAN to perform imputations on continuous features missing at random in tabular datasets (Yoon 2018b). In their algorithm GAIN, the discriminator is tasked with classifying individual variables as real or fake (imputed), as opposed to the whole ensemble. Additional input, or hint, containing the probability of each component being real or imputed is fed to the discriminator to resolve the multiplicity of optimal distributions that the generator could reproduce. The model performs considerably better than five state-of-the-art benchmarks. The GAIN algorithm was later adapted to also handle categorical features using fuzzy binary encoding, the same technique employed in HealthGAN (Yale 2019)Data augmentationThe distribution estimated by a generator model can compensate for lack of diversity in a real sample, essentially filling in the blanks in a manner comparable to data imputation. In such cases, data sampled from this distribution has the potential to help improve generalization in training predictive models. We find evidence of this by way of generating unobserved counterfactual outcomes (Yoon 2018a), or generating neighboring samples to help generalization in predictors (Che 2017). The RBM developed by Fisher et al. enabled them to simulate individualized patient trajectories based on their base state characteristics. Due to the stochastic nature of the algorithm, generating a large number of trajectories for a single patient can provide new insights of the influence of starting conditions on disease progression or quantify risk (Fisher 2019).
        
        Results: Model validation and data evaluation
To asses the solution to a generative modelling problem, it is necessary to validate the model obtained, and subsequently to verify its output. GANs aim to approximate a data distribution PP​, using a parameterized model distribution QQ​ (Borji 2018). Thus, in evaluating the model, the goal is to validate that the learning process has led to a sufficiently close approximation. Approaches to evaluation can be categorized as either quantitative or qualitative. 
Qualitative evaluation
The qualitative evaluation approaches found in the literature are mainly preference judgement, discrimination tasks, clinician evaluation (Borji 2018). Participants, such as medical professionals, discriminate between real and synthetic instances (Choi 2017b), or asked to rank the quality of real and synthetic samples on a numerical scale and their significance is determined with a Mann-Whitney U test (Beaulieu-Jones 2019). Similarly, visual inspection of statistics or projections of the data can help get a better understanding of model behaviour (Beaulieu-Jones 2019, Che 2017), but are often weak indicators of model performance without more objective  metrics (Jackson 2019). 
Quantitative evaluation
Comparing distributions
Numerous statistical metrics have been proposed or explored to compare the distributions of real and synthetic data (Borji 2018). We present here those employed in the publication included in the review in Tab. 2.

\begin{table}
    \caption{Metrics employed to validate trained models based on the comparison of distributions.\label{tab:evaldist}} 
        
    \begin{tabular}{@{} p{0.2\textwidth} p{0.2\textwidth} p{0.2\textwidth} p{0.2\textwidth} @{}}\toprule
        
        Metric & Description & Example & References\\\midrule
        
        Kullback-Leibler (KL) divergence & Compares the distributions isolated features by measuring the similarity of their marginal probability mass functions (PMF). & - &
        \cite{Goncalves2020}\\
        
        Maximum Mean Discrepancy (MMD) & 
        Checks the dissimilarity between the real and synthetic probability distributions using samples drawn independently from each other. & - &
        \cite{esteban2017real}\\
        
        2-sample test (2-ST) & Answers whether two samples, the real and synthetic, originate from the same distribution through the use of a statistical test. & 
        Kolmogorov-Smirnov (KS) & 
        \cite{Fisher2019,Baowaly2019}\\
        
        Distribution of Reconstruction Error & 
        Determine if the samples in the synthetic set are more similar to those in the training set than those in the testing set. & Nearest-neighbor &
        \cite{esteban2017real}\\
        
        Latent projection distribution & 
        Compares the distribution of real and synthetic samples projected back into the latent space & Mean of the variance & \cite{Zhang2020-wp}\\
        
        Domain specific measures & Comparison of the distributions according to a domain specific measure & Quantile-Quantile (Q-Q) plot (point-processes) & \cite{Xiao2017-lh}\\
        
        Classifier accuracy & Accuracy of a classifier trained to discriminate real from synthetic units. & - & \cite{Fisher2019,walsh2020generating}\\\bottomrule
        
    \end{tabular}
\end{table}

Statitistical fidelityA substitute to directly assessing the ability of the model to replicate the distribution of real data is to compare the information content or the real data against that of synthetic data. In other words, a statistical utility metric measures the value of the work that can be done with synthetic data. Primarily, authors attempt by various measures to determine if the statistical properties of the synthetic data distribution correspond to the the real distribution. These metrics are presented in Table ???. In general, statistical metrics do not offer convincing support for the quality of the synthetic data, they are often ambiguous or can be found to be misleading upon further investigation. Given the complexity of health data, low-dimensional transformations are unlikely to paint a full picture. Authors often state that no single metric taken on its own was sufficient, and that a combination of them allowed deeper understanding of the data. Synthetic data utility While utility-based metrics often provide a more convincing indicator of data realism, they mostly lack the interpretability that some statistical metrics allow. Methods aimed at evaluating the work that can be done with synthetic data are presented in Table 4. We divided these into two categories, those in which the task is of a more conceptual nature (Data utility metrics), and those based on tasks with real-world application (Application utility metrics). Note that this distinction is not based on a rigororous definition, but serves to facilitate understanding.

\begin{table}
    \caption{Metrics of data realism employing methods and measures based on evaluating the statistical properties of the synthetic data distribution, mostly in comparison with the distribution of real data\label{tab:statmetrics}} 
    
    \begin{tabular}{@{} p{0.2\textwidth} p{0.2\textwidth} p{0.2\textwidth} p{0.2\textwidth} @{}}\toprule
        Metric & Description & References\\ \midrule
        Dimensions-wise distribution (DWD) & A generative model is trained on the real data to generate a dataset of the same size. The Bernoillli success probability is compared between both datasets for each feature. & \cite{Beaulieu-Jones2019-ct,choi2017generating,chin2019generation,yan2020generating,Baowaly2019,Baowaly_2019,ozyigit2020generation}\\
        Interdimensional correlation & Dimenion-wise Pearson coefficient correlation matrices for both real and synthetic data are compared. & \cite{Beaulieu-Jones2019-ct, Goncalves2020}\cite{torfi2019generating,Frid_Adar_2018,Yang_2019,ozyigit2020generation}\\
        First-order proximity metric & {} & \cite{Zhang2020-wp}\\
        Log-cluster metric & {} & \cite{Goncalves2020}\\
        Support coverage metric & {} & \cite{Goncalves2020}\\
        Time-lagged correlations and covariates & {} & \cite{Fisher2019,walsh2020generating}\\
        Latent Space Representation (LSR) & {} & \cite{yan2020generating}\\
        Distribution of Jaccard similarity & {} & \cite{ozyigit2020generation}\\
        \bottomrule
    \end{tabular}
\end{table}

\begin{table}
        
        
        \caption{Metrics of data realism employing methods and measures based on evaluating the utility of the synthetic data on practical tasks.}\label{tab:aug-metrics}
        
        \begin{tabular}{@{} p{0.2\textwidth} p{0.2\textwidth} p{0.2\textwidth} @{}} \toprule
        Metric & Description & References\\ \midrule
        
        \multicolumn{3}{Y}{\textbf{Data utility metrics}}\\ \midrule
        
        Dimension-wise prediction (DWP) & Each variable is in turn chosen as the prediction target label and the remaining as features. Two predictors are trained to predict the label, one from the synthetic data and another from a portion of the real data. Their performance is compared on the left out real data.  & \cite{choi2017generating,Camino2018-re,Goncalves2020,yan2020generating}\\[20pt]
        
        Association Rule Mining (ARM) & & \cite{Baowaly2019,Bae2020,yan2020generating}\\[20pt]
        
        Discriminative Siamese architecture & & \cite{torfi2019generating}\\[20pt]
        
        Train on synthetic, test on real (TRTS) & Accuracy on real data of some form of predictor trained on synthetic data \cite{Beaulieu-Jones2019-ct}. Correlation between important features (RF) and model coefficients (LR and SVM) \cite{Beaulieu-Jones2019-ct}. & \cite{esteban2017real,Xu2019-ay,Yoon2018-dm,chin2019generation}\\
        
        Accuracy on synthetic data of some form of predictor trained on real data & & \cite{Bae2020}\\
        
        Forward prediction accuracy of conditional generative model &
        Models trained to make forward predictions from past observations or from real data transformed with a known function can simply be evaluated for accuracy. & \cite{Xiao2018-aj,mcdermott2018semi,yoon2018gain,Yang_2019b}\\
        

        \multicolumn{3}{Y}{\textbf{Applied utility metrics}}\\ \midrule

        
        Data augmentation & A predictor is trained on a combination dataset of real and synthetic data and performance is compared with the same predictor trained on real data alone. & \cite{Yoon2018-mo}\\
        
        Predictor augmentation & The trained generative model is incorporated into a predictor's activation function by generating an ensemble of proximate data points for each instance, thereby improving generalization. & \cite{Che_2017}\\
        
        \bottomrule
        
        \end{tabular}
\end{table}

\subsection{Alternative evaluation}
In their publications, Yale et al. propose refreshing approaches to evaluating the utility of synthetic data. For example, they organized a hack-a-thon type challenge. During the event, students were tasked with creating classifiers, while provided only with synthetic data \cite{Yale_2020}. They were then scored on the accuracy of their model in real data. Similarly, in a different evaluation experiment, they attempted (successfully) to recreate published medical papers based on the MIMIC dataset using only data generated from their model HealthGAN. The implications of these results for exploratory data analysis, reproducibility experiments in cases where data cannot be distributed and more generally education in health-related scientific training are glaring. In a subsequent paper, the authors evaluate the performance of their model against traditional privacy preservation methods by using the trained discriminator component of HealthGAN to d
        
        \section{Privacy Preservation}
To evaluate the risk of reidentification of synthetic data in the publications included, empirical analyses of privacy preservation are conducted according to the definitions of Membership Inference (MIA), Attribute Disclosure (AD)  \cite{choi2017generating,Goncalves2020,yan2020generating} and Reproduction rate \cite{Zhang2020-wp}. Cosine similarities between pairs of samples are also employed \cite{torfi2019generating}. All studies report low success rates for these types of attacks, while there is little effect from the sample size. Broadly, an MIA attack aims to determine if a particular record was used to train a machine learning model \cite{chen2019ganleaks}. There is no canonical process by which an attack is conducted, nor specification of the data assets initially in possession of the attacker. For a comprehensive taxonomy of MIA against GANs, refer to the suitably titled publication by Chen et al. in which medGAN was subjected to a number of trials.\par
In black-box and white-box type attacks, including the LOGAN \cite{hayes2017logan} method, medGAN performed considerably better than WGAN-GP \cite{gulrajani2017improved}, the algorithm which served as basis for improvements to medGAN in publications discussed in Section 3.1. Overall, the authors note that releasing the full model poses a high risk of privacy breaches and that smaller training sets (under 10k) also lead to a higher risk.\par  
AD is defined as the risk of an attacker correctly infering unknown attributes of a patient's record, given a number of known attributes. Goncalves et al. evaluated MC-medGAN against multiple non-adversarial generative models in a variety of privacy compromising attacks, including AD, obtaining inconsistent results for MC-medGAN \cite{Goncalves2020}. While this is not mentioned by the authors, multiple results reported in the publication point to the fact that the GAN was not properly trained or suffered mode-collapse.\par
Numerous attempts have been made to confer traditional privacy guarantees that deteriorate data, such as differentially-private stochastic gradient descent. By limiting the gradient amplitude at each step and adding random noise, AC-GAN could produce useful data witth $\epsilon=3.5$ and $\delta<10^{-5}$ according to the definition of differential privacy \cite{Beaulieu-Jones2019-ct, esteban2017real,chincheong2020generation}. \cite{BaeAnomiGAN2020}. Uniquely, Bae et al. ensure privacy with a probabilistic scheme that ensure indistinguishably, but also maximizes utility. Specifically, a multiplicative perturbation by random orthogonal matrices with input entries of $k × m$ medical records and a second second discriminator in the form of a pretrained  predictor \cite{Bae2020}. Means to confer privacy guarantees on synthetic data generated by GANs are being actively researched in a variety of fields, many of which are a priori readily applicable to health data. At this stage, however, contradictory results have between obtained where the statistical fidelity of the synthetic seemed to be preserved, but utility-based measures based on a classification were degraded by incorporating DP.\par
\subsection{The status of fully synthetic in regards to current privacy regulations}
It seems intuitively possible that the artificial nature of synthetic data essentially prevents associations with real patients, however the question is never directly addressed in the publications. An extensive Stanford Technological Review legal analysis of synthetic data concluded that laws and regulations should not treat synthetic data indiscriminately from traditional privacy preservation methods \cite{bellovin2019privacy}. They state that current privacy statutes either outweigh or downplay the potential for synthetic data to leak secrets by implicitly including it as the equivalent of anynonymization. 
\subsection{GAN-centric approach to privacy}
Some have put forward the notion that preventing overfitting and preserving privacy may not be conflicting goals \cite{Wu2019-ui,Mukherjee2019-vu}. In privGAN, Mukherjee et al., an adversary is introduced, forcing the generator to produce samples that minimize the risk of MIA attack, in addition to cheating the discriminator. The combination of both goals has the explicit effect of preventing overfitting, and their algorithm produces samples of similar quality to non-private GANs.\par
The discordance between the theoretical concepts of DP, which are  based ultimately on infinite samples, and the often insufficient data on which the probability of disclosure is calculated remains deficient. Therefore, Yoon et al. have postulated an intriguing alternative view of privacy \cite{Yoon2020}. They propose to emphasize measuring identifiability of finite patient data, rather than the probabilistic disclosure loss of DP based on unrealistic premises. Simplistically, they define identifiability as the minimum closest distance between any pair of synthetic and real samples. In their implementation, the generator receives both the usual random seed and a real sample as input. This has the effect of mitigating mode collapse, but also of reproducing the real samples. On the other hand, the discriminator is equipped with an additional loss metric based on a measure of similarity between the original sample and the generated one, thus ensuring the tuneable threshold of identifiability is met. Their results on a number of previously discussed evaluation metrics are encouraging.\par
In a similar approach, Yale et al. broke away from the theoretical guarantees of traditional methods with a measure native to GANs. Their proposal is a metric quantifying the loss of privacy, a concept more aligned with the objective of GANs to minimize the loss of data utilit \cite{yale:hal-02160496,p2019}. They point out, quite appropriately, the advantage of concrete measureable values of loss in utility and privacy when making the decision of releasing sensitive data. Briefly, the Nearest Neighbor Adversarial Accuracy measures the loss in privacy based on the difference between two nearest neighbor metrics. The  first component is the proportion of synthetic samples that are closer to any real sample than any pair of real samples. The second component is the reverse operation. In a subsequent paper, HealthGAN evaluated against traditional privacy preservation methods with a variant of the IA based on the nearest neighbor metric. HealthGAN performs considerably better than all other methods, while still maintaining utility on a prediction task .


       




    \section{Discussion}
\subsection{Applications of GANs for health data and innovation}

Overall, the published \gls{gan} algorithms for \gls{ohd} provided equivalent or superior performance against the statistical modeling-based methods against which they were benchmarked. Importantly, their capabilities are highly relevant to the medical field: domain translation for unlabeled data, conditional sampling of minority classes, data augmentation, learning from partially labeled or unlabeled data, data imputation, and forward simulation of patient profiles. While some of these claims are overoptimistic or lack convincing evidence, they paint an encouraging picture for the value of synthetic \gls{ohd} and the transformative effect it could have on healthcare initiatives and scientific progress. In this regard the replication of medical studies with synthetic data by \citeauthor{Yale_2020} substantiate the value of \gls{sd} for exploratory data analysis, reproducibility on restricted data and more generally education in scientific training are glaring. 
\begin{quote}
\textsc{}
\end{quote}

\begin{tcolorbox}[sharp corners, colback=teal!10, colframe=teal!80, title=Data utility]
More research efforts should by directed to demonstrating that synthetic data generated by GANs posses sufficient utility for scientific analyses. Reproducing published reslults of medical research is a straightforward and convincing way to acheive this. 
\end{tcolorbox}

\subsection{Challenges posed by OHD}
The challenges posed by health data are obvious, and a number of recurrent factors influenced the outcome of efforts to develop \glspl{gan} for \gls{ohd}. These problems are not limited to generative algorithms, but also \gls{ml} in general. While the progress in developing new algorithms has great momentum, their application and adoption will undoubtedly be more sluggish, as has been the case with predictive \gls{ml}.\par

In the case of generative models, multi-modality is one aspect that caused the most trouble in achieving a stable training procedure. At the outset, preventing mode collapse was an issue that attracted the most research efforts, in addition to data containing combinations of categorical and continuous features. A rapid succession of efforts aimed at improving \gls{medgan} by incorporating the latest machine learning techniques, known to improve performance across a broad range of applications, showed continued improvements. However, taken as a whole the efforts were haphazard and often yielded unsurprising results. This is not unexpected in a new field, and more concerted efforts to systematically approach the problems would surely formalize the research.\par

While the problem of mode collapse has been alleviated, evidence has yet to be provided with regards to ensuring that the finer details of the distribution are estimated with sufficient granularity to produce realistic patient profiles. In this direction conditional training methods have led to improvements. For example, when labels corresponding to sub-populations or classes are used to condition the generative process. \citeauthor{Zhang2020} showed that conditioned training with categorical labels, in this case age ranges, improves utility for small datasets, but not with larger samples \cite{Zhang2020} As described in Section \ref{noauto}, \gls{heterogan} further introduces constraint-based loss. Based on the distribution of individual features and utility-based metrics, the authors argue that the bias intrinsic to their methods has not led to undesirable bias or side-effects in other aspects of the learned distribution. 

The idea of introducing human knowledge in the otherwise naive training process has gained some attention. Not only can this improve the speed and quality of training, but also implies some degree of interpretability.

\subsection{Evaluation metrics and benchmarking}
In regards to the practices of evaluation, the choice of optimal metrics and indicators is still being explored. Overall, no evaluation metric proposed addresses the concept of realism in synthetic data. The blatant observation is that the efforts are far from consistent or systematic. This has led to a number of issues. As a striking example, competing methods are often compared with different metrics or with contradictory results in different datasets \cite{baowaly_2019_IEEE,baowaly_2019_jamia,Camino2018-re,Choi2017-nt,Zhang2020}. In their evaluation of \gls{medgan}, Yale et al. argue that the positive resemblance of plotted feature distribution of synthetic data against real data is due to the fact that the model's architecture tends to favor reproducing the means and probabilities of each diagnosis column. For example, synthetic data contains samples with an unusually high number of codes. Their hypothesis is that these samples are used by the algorithm to discharge the rare medical codes with weak correlation to balance the distributions. However, they stated in their experiments that comparing \gls{pca} plots of real and synthetic data for various generation methods was insightful to get an impression of their behavior \cite{Yale_2020}.\par
Qualitative evaluation, in its current form, provides little evidence. For medical experts, these representations are meaningless. As such, the results of qualitative evaluation often state that synthetic data is indistinguishable from the real data \cite{Choi2017-nt,Wang_2019}. It is doubtful that they could in fact be. Esteban et al. found that participants avoided the median score and were not confident enough to choose either extreme (Esteban 2017).\par
Reproducing aggregate statistical properties is rather unconvincing evidence that a model has learned to reproduce the complexity of patient health trajectories. Choi et al. found that although the synthetic sample seemed statistically sound, it contained gross errors such as gender code mismatches and suggested the use of domain-specific heuristics \cite{Choi2017-nt}. \gls{heterogan} was an encouraging step in this direction, but it may be difficult to scale. In some cases the statistical metrics may be contradictory, such as when the ranking of medical frequencies are wrong, but the data augmentation leads to improved performance \cite{Che_2017}
Utility-based metrics provide a more solid evaluation of data quality. However, these metrics only confirm the value of the data according to a narrow context. They are indicative of realism so far as a patient's state is indicative of a medical outcome. Moreover, they do not provide any insight about the validity of the relations found in a patient record and its overall consistency. 

\subsection{Analysis of OHD-GAN}
\subsubsection{Data representation and algorithm architecture}
We observed that majority of methods included in the review made use of  altered representations of patient records. Namely, through feature engineering the data is transformed from its original form. This is in part due to the inconvenient properties of health data, such as missingness. However, it is somewhat apparent that the main motive is to accommodate existing algorithms. Along with demographic variables, \gls{ohd} data mostly takes the form of triples composed by a timestamp, a medical concept and the recorded value. Their count is different for each patient, irregular intervals between each triple and the number of possible values in a dimensions can be huge. Moreover, there are generally multiple episodes of care, each with a different cause. The form and content is not typically considered practical for machine learning. \par
At varying degrees, depending on the transformations, information is being lost or bias is being  introduced. For example, when data are reduced by aggregation to one-hot encoding of binary or count variables, the complex relationships found in medical data are, for the most part, lost. Similarly, information is lost when forcing continuous time-series into a regular representation, by truncating, padding, binning or imputation. Moreover, it is highly unlikely that the data is missing at random, introducing the potential for bias when a large part of the real data is rejected on this basis. Truncating the medical codes to their parent generalizations \cite{Zhang2020, Choi2017-nt}.  In brief, loss of information content is being preferred by molding and discarding arbitrarily the data to the benefit of performance metrics, as opposed to the more tricky alternative of developing algorithms according the data.\par
Deep architectures are based on the intuition that multiple layers of nonlinear functions are needed to learn complicated high-level abstractions \cite{Bengio_2009}. CNN capture patterns of an image in a hierarchical fashion, such that in sequence, each layer forms a representation the data at a higher level of abstraction. This type of data-oriented architecture has led to impressive performance for CNN and image data. The same principle can be applied to health data. An algorithm developed in a hierarchical structure, was demonstrated to form representations of \gls{ehr} that capture the sequential order of visits and co-occurrence of codes in a visit have led to improved predictor performance, and also allowed for meaningful interpretation of the model \cite{choi2016multi}. Similarly, models of time-series based on a continuous time representation, such as found in \gls{ehr} data, have shown improved accuracy over discrete time-representations \cite{rubanova2019latent,de2019gru}. Nonetheless, creative adaptations of the data for existing architectures have provided surprising results. For example, \gls{ohd} input into a CNN were transformed to image(bitmaps) in which the pixels encoded the information \cite{Fukae2020}.

\section{Recommendations}\label{sec:recommend}
\subsection{Basic models}\label{sec:basic}

Overall, evaluation methods were superficial or uni-dimensional. Finding convincing and robust evaluation metrics for synthetic health data is an open issue. Even more so when the learning task is poorly defined or the scope of the problem is too large. The difficulty of explaining or validating the realism of data representing a patient, often longitudinal and which factors deferentially contribute to disease characterization makes the assessment of synthetic data ambiguous, thus demanding stronger evidence to claims.\par
Modelling efforts for \gls{ohd}-GAN should be limited in scope to a single data type or modality. This is favourable for a number of evaluation related aspects. Firstly, it makes qualitative evaluation by visual inspection from experts possible and meaningful. Secondly, for same reasons, the behaviour of the model can be assessed straightforwardly. The generative process can be influenced intentionally to observe the effect on the properties of the output. Finally, it allows for quantitative evaluation with domain specific metrics. The scope should clearly identify the purpose of the data generation, its utility and the target patients\cite{Capobianco2020,Kappen_2016, Kappen_2016a}

\subsection{Data-driven architecture}\label{sec:archi}
The algorithm architecture of \gls{ohd}-GAN should be engineered to match the process that generated the data, not the other way around. Data should be used and generated in the form it is first collected. In addition to preventing information loss, this ensures models will reflect the real generative process. Such models are more likely to provide insights into the system they are taught to imitate and further our understanding about them. Furthermore, the learned statistical distribution is inevitably more meaningful and interpretable, facilitating applications in the healthcare domain and supporting the inference of insights from the learned model parameters.
\subsection{Interpretability}
Even though a few authors explored the behavior of their models according to various methods, the subject was left largely unmentioned. It is imperative that future experimentation and publication give equal importance to evaluating the interpretation of their models and means to do so, as for performance. In the healthcare domain, black box machine learning models find little adoption, and synthetic data is most often met with attacks to its validity.

\section{Directions for future research}
\subsection{Building a patient model}
The ultimate goal for generative models of \gls{ohd} must be to develop an algorithm capable of learning an all encompassing patient model. It would then be possible to generate full \gls{ehr} records on demand, integrating genetic, lifestyle, environmental, biochemical, imaging, clinical information into high-resolution patient profiles \cite{Capobianco2020}. This is in fact the intention of the patient simulator Synthea. However, Synthea will eventually face a problem with scalability and the capacity of semi-independent state-transition models to coordinate in capturing long-range correlations.\par

Once basic models of health data, as described in Section \ref{sec:basic}, have been developed and validated, these can be progressively combined in a modular fashion to obtain increasingly complex patient simulators. Furthermore, having designed the architecture of these basic models on the underlying data in a way that is comprehensible, as described in \ref{sec:archi}, will facilitate the composition of more complex models. Inputs, outputs and parts of these models can be conditionally attached to others such that the generative process occurs in a way that reflects the real generative process.

\subsection{Evaluating complex patient models \label{sec:evaluation-cqm}}
Once more complex models are developed, the problem is again finding meaningful evaluation metrics of data realism. Capobiano et al. insist on the necessity for data performance metrics encompassing diagnostic accuracy, early intervention, targeted treatment and drug efficacy \cite{Capobianco2020}. In their publication exploring the validation of the data produced by Synthea, Chen et al. provide an interesting idea to achieve this \cite{Chen_2019}. Noting that the quality of care is the prime objective of a functional healthcare system, they suggest using \glspl{cqm} to evaluate the synthetic data. These measures "are evidence-based metrics to quantify the processes and outcomes of healthcare", such as "the level of effectiveness, safety and timeliness of the services that a healthcare provider or organization offers."(Chen 2019). High-level indicators such as \glspl{cqm} domain specific measures of quality, specifically designed for higher level or multi-modal representations of healthcare data. The constraints introduced in \gls{heterogan} should be leverage to evaluate the realism of the synthetic data, rather than bias the generator training. Composing a comprehensive set of such constraints could possibly serve as a standardized benchmark.
At the individual level, Walsh et al. employ domain specific indicators of disease progression and worsening and compare agreement of the simulated patient trajectories with the factual timelines \cite{walsh2020generating}.\par
In addition to \gls{cqm}, we propose the use of the Care maps used by the Synthea model to simulate patient trajectories as evaluation metrics \cite{Walonoski_2017}. Care maps are transition graphs developed from clinician input and Clinical Practice Guidelines, of which the transition probabilities are gathered from health incidence statistics. While these allow the Synthea algorithm to simulate patient profile with realistic structure, they also prevent it from reproducing real-world variability. Conversely, while \glspl{gan} have the ability to reproduce the quirks of real data, they also lack the constraints preventing nonsensical outputs. As such, Care maps provide an ideal metric to check if the synthetic data conforms to medical processes.\par 
In fact, has been used before in a competition where participants were given synthetic data from finite state transition machines with know probabilities and tasked to build and learn models that would reproduce those of the original, unseen models. The participants according to the Perplexity metric. Commonly used in NLP, quantifies how well a probability distribution or probability model predicts a sample \cite{Verwer_2013}. We postulate that the Synthea models built with real-world probabilities would provide a unique and robust way to evaluate synthetic data according to the metric proposed above, among other means to utilize the state-transition in Synthea and their modularity.

\subsubsection{Latent space \label{sec:latent-space}}
The latent space representation, the lower-dimensional vector space of the data, can provide means for evaluation and interpretability and it's potential should be explored when developing algorithms \cite{lui2019-latent}. Numerous publications have shown that they capture meaningful properties and structure of the data, reducing complexity to a level that lends itself to interpretation \cite{Way2020, Koumakis2020}. In one instance involving transcription factor micro-array data, a close one-to-one mapping could be obtained from the last hidden layer, in addition to the higher level layers that related to biological processes in a hierarchical fashion \cite{chen2016-latentyeast}. Pushing the boundaries further, by correlating the output features of a GAN with the latent space dimensions allowed controllable semantic manipulation of the generated data \cite{Wang2020latent,Ding2020latent,Li2020latent}, or provided new insights by exploring structured perturbations \cite{lui2019-latent}.

\subsubsection{Opportunities and application to current events}
Synthetic and external controls in clinical trials are becoming increasingly popular \cite{Thorlund2020}. Synthetic controls refer to cohorts that have been composed from real observational cohorts or \gls{ehr} using statistical methodologies. While the individuals included in the cohorts are usually left unchanged, micro-simulations of disease progression at the patient level are used to explore long-term outcomes and help in the estimation of treatment effects \cite{Thorlund2020, Etzioni2002}. Synthetic data generated by \glspl{gan} could be transformative for the problem of finding control cohorts.\par
With the COVID-19 pandemic scientists have become increasingly aware of and vocal about the need for data sharing between political borders \cite{Cosgriff_2020,Becker_2020,McLennan_2020}. An obvious application is generating additional amounts of data in the early stages of the pandemic, potentially creating opportunities earlier. Synthetic is data not only an opportunity to facilitate the exchange of data, but also adjust the biases of samples obtained from different localities. Factors such as local hospital practices, different patient populations and equipment introduce feature and distribution mismatches \cite{Ghassemi2020}. These disparities can be mitigated by translation of \gls{gan} algorithms, such as \gls{cycle-gan} proposed by Yoon et al.
    \section{Sourcecode and datasets}
The algorithms presented in this review can undoubtedly find usefulness for other health data or similar problems. Most importantly they can be reevaluated on other datasets or improved by adapting them with latest ML techniques. We present in Table \ref{tab:sourcecode} a list of links to the source code published by the authors. In addition, we present in Table \ref{tab:datasets} the datasets which were employed by the authors in their experiments, for those who were referenced and available. 


\begin{table}[H]
    \caption{Source code and data released and made open-source by the authors\label{tab:sourcecode}}
    
    \begin{tabular}{@{}lllll@{}}
        Algorithm & Format & Location & Source code & Data\\ \toprule
        
        AC-GAN \cite{Beaulieu-Jones2019-ct} & Jupyter notebook & GitHub & \href{https://github.com/greenelab/SPRINT_gan}{greenelab/SPRINT\_gan} & \checkmark \\
        
        Ward2ICU \cite{severo2019ward2icu} & Python & GitHub & \href{https://github.com/3778/Ward2ICU}{3778/Ward2ICU} & \checkmark\\
        
        \gls{anomigan} \cite{BaeAnomiGAN2020} & Python, Tensorflow & Github & \href{https://github.com/hobae/AnomiGAN/}{hobae/anomigan} & \\
        
        \gls{gain} \cite{yoon2018imputation} & Python, Tensorflow & Github & \href{https://github.com/jsyoon0823/GAIN}{jsyoon0823/GAIN} & \checkmark\\
        
        \gls{rgan} \citeauthor{esteban2017real} & Python, Tensorflow & Github & \href{https://github.com/ratschlab/RGAN}{ratschlab/RGAN} & \checkmark\\
        
        \bottomrule
    \end{tabular}
\end{table}


\begin{table}[H]
    \footnotesize
    \caption{Dataset used in the publications\label{tab:datasets}}
    \begin{tabularx}{\textwidth}{@{}XX@{}}\toprule
    Dataset & Link\\\midrule
    
    SPRINT Clinical Trial Data \cite{wright2016randomized} 
      
    & \href{https://challenge.nejm.org/pages/home}{SPRINT Data Analysis Challenge}\\
    
    Coalition Against Major diseases Online data Repository for AD \cite{Neville_2015} 
     
    & \href{https://c-path.org/programs/dcc/projects/alzheimers-disease/coalition-against-major-diseases-consortium-database-camd-admci/}{Critical Path Institute (C-Path)}\\

    Philips eICU \cite{pollard2018eicu}    & \href{https://physionet.org}{Physionet \cite{Goldberger_2000}}\\
    
    Multiparameter Intelligent Monitoring in Intensive Care (MIMIC-III v1.4) \cite{Johnson_2016}   & \href{https://mimic.physionet.org}{MIMIC Physionet} \cite{Goldberger_2000}\\
    
    Vanderbilt University Medical Center Synthetic Derivative \cite{Roden_2008}   & \href{https://victr.vumc.org/biovu-description/}{BioVU}\\
    
    UC Irvine Machine Learning Repository \cite{Dua:2019}  & \href{http://archive.ics.uci.edu/ml/index.php }{UCI ML repository}\\
    
    Ward2ICU \cite{severo2019ward2icu}     & \href{https://arxiv.org/abs/1910.00752}{ArXiv}\\
    
    SEER Cancer Statistics Review (CSR) \cite{noone2018cronin}   & \href{https://seer.cancer.gov/data/access.html}{SEEr Incidence database}\\
    
    PREAGRANT \cite{Fasching_2015}  & Seemingly not publicly available. Correspondence address: \href{mailto:peter.fasching@uk-erlangen.de}{peter.fasching@uk-erlangen.de} \\
    
    New Zealand National Minimum Dataset (hospital events) \cite{events}    & \href{https://www.health.govt.nz/nz-health-statistics/access-and-use/data-request-form}{Data request form}\\
    
    Sutter Palo Alto Medical Foundation (PAMF) \todo{find more info about this data} \cite{Choi2017-nt}    &\\
    
    Heart failure study dataset from Sutter \cite{Choi2017-nt}  & \\
    
    \bottomrule
    \end{tabularx}
\end{table}
    \section{Conclusion}
\normalsize
\Gls{sd} has been a subject of interest for quite some time, with officials seeing enough value to launch longitudinal state-wide endeavours such as the Synthetic Data Project (SDP), funded by the United States Department of Education(USDOE) \cite{Bonnery2019-ug}. They dismiss a series of anonymization techniques, stating the burden on worker and financial resources, and the privacy guarantees that would not sufficient for governmental agencies. Issues that have only gained weight with the accumulation of big data, and the number of new sources growing consistently. The questions they hoped to answer at the start of the project in 2016 are still not fully answered (evaluation, scientific validity, legal implications). Their 2019 report on the experience is packed with interesting insights. Noting the distrust people tend to have of synthetic data, they were the ones who first proposed the idea conducting experiments on synthetic data, that could then be confirmed on real data by simply sending the analysis to the data holder (with the logistics described extensively and augmented by a flowchart).\par

The publication ends with a series case reports. The instances where the data could not satisfy requirements are analyzed with the aim of informing similar projects in the future. However the bulk of reports describe cases where was highly applicable. They concluded by predicting that the cost of generating \gls{sd} will diminish and that the methods to do so will improve. Their hopes for \gls{sd} include: easier access for researcher to the wealth of data, increased access providing downstream benefits at the state level, the these benefits encourage others to undertake similar projects that would increase generalizability of findings across states, and a preference for open data.

\renewcommand{\epigraphsize}{\footnotesize}
\setlength{\epigraphwidth}{12cm}
\epigraph{
    "[although some argue for] having secured data centers for administrative data utilization [...], our experience suggests that such centers may not solve the desire for fast turn-around research or broaden access to those with unique perspectives. Synthetic data represent a promising approach for increasing easy access to secure data while simultaneously protecting the confidentiality of individuals."}{\textit{Daniel Bonnéry, Yi Feng, Angela K. Henneberger, Tessa L. Johnson,\\ Mark Lachowicz, Bess A. Rose, Terry Shaw,\\ Laura M. Stapleton, Michael E. Woolley and Yating Zheng}}
    
The \gls{gan} was devised in 2014 in Montreal, Canada by \citeauthor{goodgan} at Université de Montreal. Two years before the start of the SDP, which must of been planned over a few years. It was too early for them to know about this obscure technique based on two neural networks competing against each other. Since then, \href{http://papers.nips.cc/paper/5423-generative-adversarial-nets}{Generative Adversarial Networks} \cite{goodgan} has inspired \textbf{23805} citations and algorithms capable of synthesizing data of impeccable similitude. We have surveyed a multitude of \gls{gan} algo

\pagebreak

    \pagebreak

    %\clearpage
    \printglossary[type=oalgo]
    \printglossary[type=\acronymtype]
    \printglossary

    \pagebreak

    \bibliography{biblio}

\end{document}
