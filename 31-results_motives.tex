\subsection{Motivations for developing OHD-GAN}
The authors mention a wide range of potential applications for generative models of OHD. While some of these goals are overoptimistic and have yet to be realized, they paint an encouraging picture for the value of synthetic OHD and the transformative effect it could have on healthcare initiatives and scientific progress. We briefly describe the four prevailing themes in data augmentation (Sec.\ref{sec:augmentation}), privacy and accessibility (Sec.\ref{sec:access_privacy}), precision medicine (Sec.\ref{sec:precision_med}) and  modelling simulations (Sec.\ref{sec:models_twins}). 

    \subsubsection{Data augmentation}\label{sec:augmentation}
    Data augmentation is mentioned in nearly all publications. Most commonly, reference is made to the demonstrated generalization benefit synthetic data can have in predictive algorithms by providing additional information about the real data distribution \cite{Wang_2019,Che_2017,Yoon2018-dm, Yoon2018-mo}. Similarly, domain translation and semisupervised training approaches could support predictive tasks that lack data with accurate labels, paired samples, or present class imbalance \cite{Che_2017,mcdermott2018semi}. 

    \subsubsection{Enhancing privacy and increasing data accessibility}\label{sec:access_privacy}
    SD is seen as the key to unlocking the value of OHD being unexploited due to privacy concerns. Preserving privacy can broadly be described as reducing the risk of reidentification attacks to an acceptable level. This level of risk is quantified when releasing data anonymized with differential privacy. Authors noted that highly restricted access to OHD is hindering machine learning, and more generally scientific progress \cite{Beaulieu-Jones2019-ct, Baowaly_2019,Che_2017,esteban2017real,Fisher2019}. Due to its artificial nature, SD is put forward as a means to forgo data use agreements, while potentially providing greater privacy guarantees\cite{Beaulieu-Jones2019-ct, Baowaly_2019,esteban2017real,Fisher2019,walsh2020generating}. In addition, differential private GAN training shows evidence of reducing the loss of utility. \todo Overall, enabling access to greater variety, quality and quantity of OHD could have positive effects in a wide range of fields, such as software development, education, and training of medical professionals. 
    
    \subsubsection{Enabling precision medicine}\label{sec:precision_med}
    Authors point out that the ability to conduct simulations of disease progression for individual patients could have transformative impacts on healthcare. Generative models conditioned on a patient's baseline state could help inform clinical decision making by quantifying disease progression and outcomes \cite{walsh2020generating, Fisher2019}. Additionally, stochastic simulations of individual patient profiles could help quantify risk at an unprecedented level of granularity \cite{Fisher2019}. Predicting patient-specific responses to drugs is still a new field of research, a problem known as Individualized Treatment Effects (ITE). The task of estimating ITEs is persistently hampered by the lack or paired samples, or counterfactuals \cite{Yoon2018-mo, chu2019treatment}. Various GAN algorithms were developed for domain translation, mapping a sample from its to original class to the paired equivalent. This includes bidirectional transformations, allowing GAN to learn mappings from very few, or a lack of paired samples \cite{Wolterink2017DeepMT}.
    
    \subsubsection{From patient and disease models to digital twins}\label{sec:models_twins}
    A well trained model approximates the process that generated the real information \cite{esteban2017real}. In other words, the relations learned by model, its parameters, contains meaningful information if we can learn to harness it. Interpretability is an analogous view, which is a growing field or research. Achieving models of significant complexity would open up unprecedented simulation possibilities for developing predictive systems and methods. In clinical research, such models could help quantify cause and effect, simulate different study designs, provide control samples or more generally give us a better understanding of disease progression in relation to initial conditions \cite{Fisher2019, yahi2017generative, walsh2020generating}.\par
    Approaching these ideas from above, the concept of "digital twins" represents in a way the ultimate realization of personalized medicine. A common practice in industrial sectors is high-fidelity virtual representations, or long-term simulations, of physical assets that provide comprehensive understanding of the workings, behavior and life-cycle of their real counterparts. The state of the models is continuously updated from theoretical data, real data and streaming IoT indicators. Conditionally generated input data allow the exploration of sepcific events or conditions. In a position paper, Angulo et al. draw the parallels of this technique with the current needs in healthcare and the emergence of the necessary technologies for the proposal they bring forward \cite{angulo2019towards,Angulo_2020}. The authors bring up the rapid adoption of wearables that are continuously monitoring people's physiological state. Through continuous lifelong learning, patient models inform the decisions of medical professionals, but also enable testing research hypotheses. In their proposal, GANs are an essential component of the ecosystem to ensure patient privacy and to provide bootstrap data. Notably, Fisher et al. already employ the term "digital twin" to describe their method \cite{walsh2020generating}.