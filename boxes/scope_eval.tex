\footnotesize
\tcbset{enhanced,before skip=1cm, nobeforeafter, width=0.5\linewidth}
\begin{tcolorbox}[arc=0mm, 
    colback=cadmiumgreen!10!white, 
    coltext=cadmiumgreen!90!black,  
    colframe=cadmiumgreen!90!black,
    colbacktitle=cadmiumgreen!80,
    leftrule=3mm,
    rightrule=0mm, 
    toprule=0mm, 
    bottomrule=0mm, 
    box align=top]

Modelling efforts for OHD-GAN should be limited in scope to develop robust algorithms for a single data type or modality.
\begin{itemize}
    \item This makes qualitative evaluation by visual inspection from experts possible and meaningful.
    \item The behaviour of the model can be assessed straightforwardly
    \item Conditional models are easier to develop.
    \item The evaluation metrics should not be defined solely for the purpose but from a peer-reviewed healthcare publication.
\end{itemize}

\end{tcolorbox}
\hfill
\begin{tcolorbox}[tcbox width=auto, 
    arc=0mm, 
    colback=white, 
    coltext=cadmiumgreen, 
    boxrule=0pt, 
    colframe=white,
    box align=top]
    
\epigraph{\textit{A baby learns to crawl, walk and then run.  We are in the crawling stage when it comes to applying machine learning.}}{\textit{Dave Waters}}

\end{tcolorbox}
\normalsize