 \subsection{Data Types and Feature Engineering}
    No publications made use of \gls{ohd} in its initial form, patient records in \glspl{gan} composed of many related tables (Normal form). The complexiety of a model wouuld grow rapibldy when maintaining referential inegrty and statistics between multiple tables. The hierarchy by witch these would interact with each other conditionally is no less complicated (see discussion Section \todo{section reference with a mention of a few statistical solutions that faced a number of problems}. There are however published \gls{gan} algorithms made to consume normalized database in their original form. \todo In regards to \gls{ohd}, feature engineering was used to adapt the data to task requirements, or to a promising algorithms that fit the date characteristics. The data is transformed into one of four modalities: time series, point-processes, ordered sequences or aggregates described in Fig. \ref{tab:features}
    
    


\begin{table}
\footnotesize
  \caption{Types of observational health data and features engineering}\label{tab:features}
  
  \begin{tabularx}{\textwidth}{@{}XXXX@{}} \toprule
  Type & Features and format & Challenges & Features engineering\\ \midrule
  
  \textbf{Time-series}\newline Continuous\newline Regular\newline Sporadic\newline 
  & Timestamped observations \newline Numerical, Categorical 
  & Observations often  Missing at Random (MAR) across time end dimensions
  & Imputation coupled with training \newline Regular \newline Data imputation \newline Binning in into fixed-size intervals \newline Combination of binning and imputation \\
  \textbf{Point-processes} 
  & Time intervals
  & 
  & Timestamped events transformed into the time delta between each consecutive occurrences\\
  \textbf{Ordered sequences} 
  & Ordered-vectors\newline Medical codes\newline 
  & Variable length\newline High-dimensional\todo\newline Long-tail distribution of codes 
  & Sequences are projected into a trained embedding that preserves semantic meaning according to methods borrowed from NLP\\
  \textbf{Tabular}\newline Denormalized\newline Relational
  & Continuous, categorical and ordinal variables in tabular form & Cross-correlations in a mixture of discrete features with high class imbalance and multimodal continuous features
  & Medical history is aggregated into a fixed-size vector of binary or aggregated counts of occurrences and combined with demographic features.\\
 
  
  \bottomrule
  \end{tabularx}
\end{table}