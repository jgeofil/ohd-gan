\section{Results}
    \subsection{Summary}
        We have found a total of \todo{count the number of publications} publications describing the development or adaption of \gls{gan} algorithms for \gls{ohd}, presented in Table \ref{tab:publications}. The type of data addressed in each of these publications can be generalized into one of two categories: time-dependent observations, such as time-series, or static representation in the form of feature vectors. Publications considering privacy either perform privacy evaluations of their algorithms and synthetic data, or exclusively concentrate on comparing methods about privacy.
        
        The most efforts are focused on adapting the current methods to the characteristics and complexities of OHD, of which multi-modality or non-Gaussian continuous features, heterogeneity, a combination of discrete and continuous features, longitudinal irregularity, correlation complexity, missingness or sparsity, class imbalance and noise are often cited. While these properties may pose a challenge for the development of useful algorithms, others aspects make the prospect of success highly valuable. In fact, the most cited motive to develop \gls{ohd-gan} is to cope with the often limited number of samples in medical datasets and to overcome the highly restricted access to \gls{ohd}.\par
        
        \subsection{Motives for developing \gls{ohd}-\gls{gan}}
The authors mention a wide range of potential applications for \gls{ohd-gan}. While some of these goals are overoptimistic and have yet to be realized, they paint an encouraging picture for the value of synthetic \gls{ohd} and the transformative effect it could have on healthcare initiatives and scientific progress. We briefly describe the four prevailing themes in the following sections: data augmentation (Sec.\ref{sec:augmentation}), privacy and accessibility (Sec.\ref{sec:access_privacy}), precision medicine (Sec.\ref{sec:precision_med}) and  modelling simulations (Sec.\ref{sec:models_twins}). 

    \subsubsection{Data augmentation}\label{sec:augmentation}
    
    Data augmentation is mentioned in nearly all publications. Although counter-intuitive, it is well known that \glspl{gan} can generate \gls{sd} that conveys more information about the real data distribution. Effectively, the continuous space distribution of the generator produces a more comprehensive set of data points, valid but not present in the discrete real data points. A combination of real and synthetic training data habitually leads to increased predictor performance \cite{Wang_2019,Che_2017,Yoon2018-ite, yoon2018imputation}. A more intelligible way to seize the concept from the point of view of image classification, in which it is known as invariances, perturbations such as rotation, shift, sheer and scale \cite{antoniou2017data}. Similarly, domain translation and \gls{semi-sup} training approaches could support predictive tasks that lack data with accurate labels, paired samples, or suffer from class imbalance \cite{Che_2017,mcdermott2018semi}. 

    \subsubsection{Enhancing privacy and increasing data accessibility}\label{sec:access_privacy}
    
    \gls{sd} is seen as the key to unlocking the unexploited value of \gls{ohd} due to privacy concerns. Preserving privacy can broadly be described as reducing the risk of \glspl{re-iden} to an acceptable level. This level of risk is quantified when releasing data anonymized with \gls{dp}. Authors noted that highly restricted access to \gls{ohd} is hindering machine learning, and more generally scientific progress \cite{Beaulieu-Jones2019-ct, Baowaly_2019,Che_2017,esteban2017real,Fisher2019}.\par
    
    Due to its artificial nature, SD is put forward as a means to forgo data use agreements, while potentially providing greater privacy guarantees\cite{Beaulieu-Jones2019-ct, baowaly_2019_IEEE, baowaly_2019_jamia,esteban2017real,Fisher2019,walsh2020generating}. In fact, \gls{gan} training according to \gls{dp} shows evidence of reducing the loss of utility in comparison to \gls{dp} alone. \todo{Find these citations} Overall, enabling access to greater variety, quality and quantity of \gls{ohd} could have positive effects in a wide range of fields, such as software development, education, and training of medical professionals. 
    
    \subsubsection{Enabling precision medicine}\label{sec:precision_med}
    
    The ability to conduct personalized simulations of disease progression for individual patients could have transformative impacts on healthcare. Generative models able to produce time-series trajectories conditioned on a patient's baseline state could help inform clinical decision making by quantifying disease progression and outcomes \cite{walsh2020generating, Fisher2019}. Ensembles of stochastic simulations of individual patient profiles such as those produced by gls{crmb} could help quantify risk at an unprecedented level of granularity \cite{Fisher2019}.\par
    Predicting patient-specific responses to drugs is still a new field of research, a problem known as \gls{ite}. The task of estimating \glspl{ite} is persistently hampered by the lack of paired counterfactual samples \cite{Yoon2018-ite, chu2019treatment}. In medical imaging, various \gls{gan} algorithms were developed for domain translation, mapping a sample from its to original class to the paired equivalent. This includes bidirectional transformations, allowing \glspl{gan} to learn mappings from very few, or a lack of paired samples \cite{Wolterink2017DeepMT}.
    
    \subsubsection{From patient and disease models to digital twins}\label{sec:models_twins}
    
    A well trained model approximates the process that generated the real data points \cite{esteban2017real}. In other words, the relations learned by the model, its parameters, contains meaningful information if we can learn to harness it. Interpretability is a growing field of research concerned with understanding how these learned parameters relate, and thus explaining the representations the algorithm has converged to in linking the features to the outcome.\par
    Achieving models of significant complexity would both open up unprecedented simulation capabilities, but also the chance to explore meaningful representations that would otherwise be beyond our reasoning.\par 
    In clinical research, such models could help quantify cause and effect, simulate different study designs, provide control samples or more generally give us a better understanding of disease progression in relation to initial conditions \cite{Fisher2019, yahi2017generative, walsh2020generating}.\par
    
    Approaching these ideas from above, the concept of "digital twins" represents in a way the ultimate realization of \gls{pm}. A common practice in industrial sectors is high-fidelity virtual representations of physical assets. Long-term simulations, that provide an overview and comprehensive understanding of the workings, behavior and life-cycle of their real counterparts. The state of the models is continuously updated from theoretical data, real data and streaming \gls{iot} indicators.\par
    Intently conditioned input data allows the exploration of specific events or conditions. In a position paper on the subject, Angulo et al. draw the parallels of this technique with the current needs in healthcare and the emergence of the necessary technologies for actionable models of patients. \cite{angulo2019towards,Angulo_2020}. The authors bring up the rapid adoption of wearables that are continuously monitoring people's physiological state. 
    Wearables are one of many mobile digitally connected devices that collect patient data over a broad range of physiological characteristic and behavioral patterns \cite{coravos2019developing}. This emerging trend known as \gls{dbio} has already led to studies demonstrating predictive models with the potential for improved patient care \cite{snyder2018best}. Through continuous lifelong learning, integrating  multiple modes of personal data, generative patient models could inform diagnostics of medical professionals and also enable testing treatment options. In their proposal, \glspl{gan} are an essential component of the ecosystem to ensure patient privacy and to provide bootstrap data. Notably, Fisher et al. already employ the term "digital twin" to describe their process \cite{walsh2020generating}.
        
\begin{sidewaystable}[htpb]
\scriptsize
  \centering
    \caption{Publications included}\label{tab:publications}
  
    \begin{tabularx}{\textwidth}{@{}p{3cm}XXXXXX@{}} \toprule
    Publication 
    & Algorithm 
    & Focus 
    & Data type 
    & Algorithm and Techniques 
    & Evaluation 
    & Privacy\\ \midrule
    
    \cite{Choi2017-nt}
    & \thealgo{medGAN} 
    & Discrete features, Mode collapse 
    & Binary, Ordinal 
    & \gls{ffn},\gls{ae}, \gls{mba}, \gls{bn}, \gls{sc} 
    & \gls{dwpre}, \gls{dwpro} 
    & \gls{ad}, \gls{pd}\\
    
    \cite{esteban2017real}
    & \thealgo{RGAN}, \thealgo{RCGAN} 
    & Real-valued time-series, conditional training 
    & Regulary obsverved time-series 
    & \gls{lstm}, \gls{swl}, \gls{cgan} 
    & \gls{mmd}, \gls{tstr}, \gls{trts}, \gls{auroc}, \gls{auprc}, \gls{pta} 
    & \gls{dp-sgd}\\
    
    \cite{yahi2017generative} 
    & $-$
    & Continuous time-series, Drug laboratory effects (DRE) 
    & Paired pre and post exposure time-series 
    & \algo{medGAN}, Clustering, t-SNE 
    & \gls{mse}
    & $-$ \\
    
    \cite{Che_2017} 
    & \thealgo{ehrGAN}, \thealgo{SSL-GAN} 
    & Discrete time-series, semi-supervised augmentation 
    & Sequences of medical codes 
    & 1D-CNN, Word2vec, \gls{vcd}  
    &\gls{cc}, \gls{fd}, \gls{ssa}
    & $-$\\
    
    \cite{Xiao2017-lh} 
    & \thealgo{WGANTPP} 
    & Temporal Point Processes 
    & Sporadic occurences 
    & \gls{lstm}, \gls{wgan} 
    & Poisson process 
    & \gls{qq} \\
    
    \cite{Yoon2018-dm}
    & \thealgo{RadialGAN} 
    & Multi-domain translation, features and distribution mismatch, cycle-consistency, augmentation 
    & Tabular, discrete and continuous 
    & \gls{ffn} 
    & \gls{cgan}, \gls{wgan}, \gls{md-cc} 
    & \gls{pta}, \gls{auroc}, \gls{auprc} \\
    
    \cite{Yoon2018-mo} 
    & \thealgo{GANITE} 
    & \gls{ite}, unobserved counterfactuals 
    & Feature, treatment and outcome vectors 
    & \gls{cgan} pair 
    & See publication 
    & See publication \\
    
    \cite{Camino2018-re} 
    &\thealgo{MC-ARAE}, \thealgo{MC-medGAN}, \thealgo{MC-GumbelGAN}, \thealgo{MC-WGAN-GP} 
    & Data composed of multiple categorical variables 
    & Multiple categorical variables represented as one-hot encoded vectors 
    & \algo{medGAN}, \algo{WGAN-GP}, \gls{gumbel-gan}, \gls{arae} 
    & See Section \ref{sec:categorical} 
    & $-$\\
    
    \cite{Zhang2020}
    &\thealgo{EMR-WGAN}
    & Medical codes, improving training, evaluation metrics
    & Binary vector of occurrence over the medical codes. Low-prevalence of codes by which approximately half the dataset is discarded.
    & \gls{wgan}, \gls{bn}, \gls{ln}, \gls{cgan}
    & \gls{dwpro}, \gls{dwpre}, \gls{lsr}, \gls{fop}
    &\gls{ad}, \gls{mi}, \gls{rr}\\
    
    \bottomrule    \end{tabularx}
\end{sidewaystable}
        
        \subsection{Data oriented GAN development}
\subsubsection{Auto-encoders and categorical features}
In what is, to the best of our knowledge the first attempt at developing a GAN for OHD. Choi et al. focus on the problem posed by the incompatibility of categorical and ordinal features with back-propagation. Their solution is to pretrain an Autoencoder (AE) to project the samples to and from a continuous latent space representation and retain the decoder portion to form a component of the GAN \cite{Choi2017-nt}. In the algorithm \thealgo{medGAN,} the trained decoder in incorporated into the generator and maps the randomly sampled input vectors from latent space representation back to discrete features. This first exemplar of synthetic OHD generated by GAN inspires a series of enhancements.\par

Numerous efforts were made to improve the performance of \algo{medGAN}. Among the first, Camino et al. developed \thealgo{MC-medGAN} in which they modified the AE by adding a Gumbel-Softmax \cite{jang2016categorical} activation layer after splitting its output with a dense layer for each categorical variable and finally concatenating of the Gumbel-Softmax \todo layers \cite{Camino2018-re}. The authors also developed an adaptation based on recent training techniques: Wassertein GAN (WGAN) \cite{arjovsky2017wasserstein} and a WGAN with Gradient Penalty (WGAN-GP) \cite{gulrajani2017improved}. In brief, the Wasserstein distance is a measure of distance between two probability distributions that has the property of always providing a smooth gradient. When used as the loss function of the discriminator, it generally improves training stability and mitigates mode collapse. Weight clipping is used in WGAN to ensure the discriminator lies within of 1-Lipschitz functions. The undesirable effects of weight clipping are eliminated by rather imposing a penalty on the gradient . \thealgo{MC-WGAN-GP} is the equivalent of \algo{MC-medGAN} but with Softmax layers. The authors report that the choice of a model will depend on data characteristics, particularly sparsity.\par 
Wasserstein's distance was widely adopted by subsequent authors for its compatibility with OHD. Baowaly et al. developed \thealgo{MedWGAN} also based on WGAN, and \thealgo{MedBGAN} borrowing from Boundary-seeking GAN (BGAN) \cite{hjelm2017boundaryseeking} which pushes the generator to produce samples that lie on the decision boundary of the discriminator, expanding the search space. Both led to improved data quality, in particular \algo{MedBGAN} \cite{baowaly_2019_IEEE,baowaly_2019_jamia}. In other effort, Jackson et al. tested \algo{medGAN} on an extended dataset containing demographic and health system usage information, obtaining results similar to those of the original \cite{Jackson_2019}. The \thealgo{HealthGAN} built upon WGAN-GP, but includes a data transformation method adapted from the Synthetic Data Vault \cite{Patki_2016} to map categorical features to and from the unit numerical range \cite{Yale_2020}. 
        
        \subsubsection{Forgoing the autoencoder}\label{noauto}

Suggesting the use of an \gls{ae} introduces noise, with \gls{emr-wgan}, Zhang et al. dispose of the \gls{ae} component of previous algorithms and introduce a conditional training method, along with conditioned \gls{bn} and \gls{ln} techniques to stabilise training \cite{Zhang2020-wp}. The algorithm was further adapted by Yan et al. as \gls{heterogan} to better account for the conditional distributions between multiple data types and enforce record-wise consistency. A recognized problem with \algo{medgan} was that it produced common-sense inconsistencies, such as gender mismatches in medical codes \cite{yan2020generating, choi2017generating}. In \gls{heterogan}, constraints are enforced by adding specific penalties to the loss function, such as limit ranges for numerical categorical pairs and mutual exclusivity for pairs of binary features \cite{yan2020generating}. \par

To develop \gls{ctgan}, Xu et al. presume that tabular data poses a challenge to GANs owing to the non-Gaussian multi-modal distribution of continuous columns and imbalanced discrete columns \cite{Xu2019-ay}. Their algorithm, composed of fully connected layers, was developed with adaptations to deal with both continuous and categorical features. For continuous features, it employs mode-specific normalization to capture the multiplicity of modes. For discrete features conditional training-by sampling is devised to resample discrete attributes evenly during training, while recovering the real distribution when generating data.\par

Other approaches include: \gls{corgan}, where the \gls{ae} is questionably replaced by a \gls{1d-cae} to capture neighboring feature correlations of the input vectors \cite{torfi2019generating}, and two basic feedforward networks based on Wassertein distance to evaluate the capacity of \glspl{gan} to model heterogeneous data of dense and sparse medical features \cite{chincheong2020generation} and to reproduce statistical properties \cite{ozyigit2020generation}. Reproducing physiological time-series \citeauthor{esteban2017real} used devise the \gls{rgan} and \gls{rcgan} based on \gls{lstm} to generate a regular time-series of physiological measurements from bedside monitors \cite{esteban2017real}. Curiously, the authors dismiss Wassertein's distance, stating that they did not find application in their experiments. In addition, each dimension of their time-series is generated independently from the others, where one would assume they are correlated. A considerable loss of accuracy is observed on their \gls{utility-metric}.

\subsection{Task oriented GAN development}
\subsubsection{Semi-supervised learning and conditional models}

To develop \gls{ehrgan}, an algorithm for sequences of medical codes that has the ability to produce neighbouring records of an input patient, \citeauthor{Che_2017} combine an Encoder-Decoder \gls{cnn} \cite{ranzato2007unsupervised} with \gls{vcd} \cite{Che_2017}. The \gls{ehrgan} generator is trained to decode a random vector mixed with the latent space representation of a particular patient. In a semi-supervised learning approach, the trained \gls{ehrgan} model is then incorporated into the loss function of a predictor where it can help generalization by producing neighbors for each input sample. Semi-supervised learning approaches are commonly employed to augment the minority class in imbalanced datasets, such as \gls{self-training} and \gls{co-training}.  Yang et al. improve on this type of approach by incorporating a GAN in the procedure \cite{yang}. The GAN is first trained on the labelled set and used to rebalance it. The standard iterative process involving the classifier ensemble is then executed until expansion ceases. As a final step, the GAN is trained on the expanded labelled set to generate an equal amount of augmentation data. The authors obtained improved performance in a number of classification tasks and multiple tablular datasets with their method.Correcting bias with domain translationTo address the heteogeneity of healthcare data from different sources, Yoon et al. combines the concepts of cycle-consistent domain translation from Cycle-GAN (Zhu 2017) and multi-domain translation from Star-GAN (Choi 2017a) to build RadialGAN to translate heterogeneous patient information from different hospitals, correcting features and distribution mismatches (Yoon 2018). An encoder-decoder pair per data endpoint is trained to map records to and from a shared latent representation. Individualized treatment effectsThe task of estimating Individualized Treatment Effects (ITE), the response of a patient to a certain treatment given a set of charaterizing features is an ongoing problem. This is due mainly to the fact that counterfactual outcomes are never observed or that treatment selection is highly biased (Yoon 2018a, McDermott 2018, Walsh 2020). In this regard, Yoon et al. employ a pair of GANs, named Generative Adversarial Nets for inference of Individualized Treatment Effects (GANITE), one for counterfactual imputation and another for ITE estimation (Yoon 2018a). The former captures the uncertainty in unobserved outcomes by generating a variety of conterfactuals. The output is fed to the latter, which estimates treatment effects and provides confidence intervals. With Cycle Wasserstein Regression GAN (CWR-GAN), a joint regression-adversarial model, McDermott et al. demonstrated a semi-supervised approach also inspired by Cycle-GAN to leverage large amounts of unpaired pre/post-treatment time-series in ICU data for the estimation of ITE on physiological time-series (McDermott 2018). The algorithm has the ability to learn from unpaired samples, with very few paired samples, to reversibly translate the pre and post-treatment physiological series. Chu et al. approach the problem of data scarcity for ITEs by designing ADTEP, an algorithm that can maximize use of the large volume of EHR data formed by triples of non-task specific patient features, treatment interventions and treatment outcomes (Chu 2019). The ADTEP algorithm they developed learns representation and discriminatory features of the patient, and treatment data by training an \gls{ae} for each pair of features. In addition to \gls{ae} reconstruction loss, a second model is tasked with identifying fake treatment feature reconstructions. Finally, a fourth loss metric is calculated by feeding the concatenated latent representations of both \gls{ae} to a logisitic regression model aimed at predicting the treatment outcome (Chu 2019). In the form of an ITE task, Wang et al. demonstrated an interesting algorithm to generate a time series of patient states and medication dosages using \gls{lstm}. In contrast to RGAN and RCGAN, in Sequentially Coupled Generative Adversarial Network (SC-GAN), patients state at the current timestep informs the concurrent medication dosage, which in turn affects the patient state in the upcoming timestep (Wang 2019). SC-GAN overcame a number of baselines on both statistical and utility metrics. Data Imputation with GANsGANs are naturally suited for data imputation, and could provide a new approach to deal with the problems of health data relating to sparsity. Statistical models developed for the multiple imputation problem increase quadraticly in complexity with the number of features, while the expressiveness of deep neural networks can model all features with missing values simultaneously efficiently. In that regard, Yoon et al. adapted the standard GAN to perform imputations on continuous features missing at random in tabular datasets (Yoon 2018b). In their algorithm GAIN, the discriminator is tasked with classifying individual variables as real or fake (imputed), as opposed to the whole ensemble. Additional input, or hint, containing the probability of each component being real or imputed is fed to the discriminator to resolve the multiplicity of optimal distributions that the generator could reproduce. The model performs considerably better than five state-of-the-art benchmarks. The GAIN algorithm was later adapted to also handle categorical features using fuzzy binary encoding, the same technique employed in HealthGAN (Yale 2019)Data augmentationThe distribution estimated by a generator model can compensate for lack of diversity in a real sample, essentially filling in the blanks in a manner comparable to data imputation. In such cases, data sampled from this distribution has the potential to help improve generalization in training predictive models. We find evidence of this by way of generating unobserved counterfactual outcomes (Yoon 2018a), or generating neighboring samples to help generalization in predictors (Che 2017). The RBM developed by Fisher et al. enabled them to simulate individualized patient trajectories based on their base state characteristics. Due to the stochastic nature of the algorithm, generating a large number of trajectories for a single patient can provide new insights of the influence of starting conditions on disease progression or quantify risk (Fisher 2019).
        
        Results: Model validation and data evaluation
To asses the solution to a generative modelling problem, it is necessary to validate the model obtained, and subsequently to verify its output. GANs aim to approximate a data distribution PP​, using a parameterized model distribution QQ​ (Borji 2018). Thus, in evaluating the model, the goal is to validate that the learning process has led to a sufficiently close approximation. Approaches to evaluation can be categorized as either quantitative or qualitative. 
Qualitative evaluation
The qualitative evaluation approaches found in the literature are mainly preference judgement, discrimination tasks, clinician evaluation (Borji 2018). Participants, such as medical professionals, discriminate between real and synthetic instances (Choi 2017b), or asked to rank the quality of real and synthetic samples on a numerical scale and their significance is determined with a Mann-Whitney U test (Beaulieu-Jones 2019). Similarly, visual inspection of statistics or projections of the data can help get a better understanding of model behaviour (Beaulieu-Jones 2019, Che 2017), but are often weak indicators of model performance without more objective  metrics (Jackson 2019). 
Quantitative evaluation
Comparing distributions
Numerous statistical metrics have been proposed or explored to compare the distributions of real and synthetic data (Borji 2018). We present here those employed in the publication included in the review in Tab. 2.

\begin{table}
    \caption{Metrics employed to validate trained models based on the comparison of distributions.\label{tab:evaldist}} 
        
    \begin{tabular}{@{} p{0.2\textwidth} p{0.2\textwidth} p{0.2\textwidth} p{0.2\textwidth} @{}}\toprule
        
        Metric & Description & Example & References\\\midrule
        
        Kullback-Leibler (KL) divergence & Compares the distributions isolated features by measuring the similarity of their marginal probability mass functions (PMF). & - &
        \cite{Goncalves2020}\\
        
        Maximum Mean Discrepancy (MMD) & 
        Checks the dissimilarity between the real and synthetic probability distributions using samples drawn independently from each other. & - &
        \cite{esteban2017real}\\
        
        2-sample test (2-ST) & Answers whether two samples, the real and synthetic, originate from the same distribution through the use of a statistical test. & 
        Kolmogorov-Smirnov (KS) & 
        \cite{Fisher2019,Baowaly2019}\\
        
        Distribution of Reconstruction Error & 
        Determine if the samples in the synthetic set are more similar to those in the training set than those in the testing set. & Nearest-neighbor &
        \cite{esteban2017real}\\
        
        Latent projection distribution & 
        Compares the distribution of real and synthetic samples projected back into the latent space & Mean of the variance & \cite{Zhang2020-wp}\\
        
        Domain specific measures & Comparison of the distributions according to a domain specific measure & Quantile-Quantile (Q-Q) plot (point-processes) & \cite{Xiao2017-lh}\\
        
        Classifier accuracy & Accuracy of a classifier trained to discriminate real from synthetic units. & - & \cite{Fisher2019,walsh2020generating}\\\bottomrule
        
    \end{tabular}
\end{table}

Statitistical fidelityA substitute to directly assessing the ability of the model to replicate the distribution of real data is to compare the information content or the real data against that of synthetic data. In other words, a statistical utility metric measures the value of the work that can be done with synthetic data. Primarily, authors attempt by various measures to determine if the statistical properties of the synthetic data distribution correspond to the the real distribution. These metrics are presented in Table ???. In general, statistical metrics do not offer convincing support for the quality of the synthetic data, they are often ambiguous or can be found to be misleading upon further investigation. Given the complexity of health data, low-dimensional transformations are unlikely to paint a full picture. Authors often state that no single metric taken on its own was sufficient, and that a combination of them allowed deeper understanding of the data. Synthetic data utility While utility-based metrics often provide a more convincing indicator of data realism, they mostly lack the interpretability that some statistical metrics allow. Methods aimed at evaluating the work that can be done with synthetic data are presented in Table 4. We divided these into two categories, those in which the task is of a more conceptual nature (Data utility metrics), and those based on tasks with real-world application (Application utility metrics). Note that this distinction is not based on a rigororous definition, but serves to facilitate understanding.

\begin{table}
    \caption{Metrics of data realism employing methods and measures based on evaluating the statistical properties of the synthetic data distribution, mostly in comparison with the distribution of real data\label{tab:statmetrics}} 
    
    \begin{tabular}{@{} p{0.2\textwidth} p{0.2\textwidth} p{0.2\textwidth} p{0.2\textwidth} @{}}\toprule
        Metric & Description & References\\ \midrule
        Dimensions-wise distribution (DWD) & A generative model is trained on the real data to generate a dataset of the same size. The Bernoillli success probability is compared between both datasets for each feature. & \cite{Beaulieu-Jones2019-ct,choi2017generating,chin2019generation,yan2020generating,Baowaly2019,Baowaly_2019,ozyigit2020generation}\\
        Interdimensional correlation & Dimenion-wise Pearson coefficient correlation matrices for both real and synthetic data are compared. & \cite{Beaulieu-Jones2019-ct, Goncalves2020}\cite{torfi2019generating,Frid_Adar_2018,Yang_2019,ozyigit2020generation}\\
        First-order proximity metric & {} & \cite{Zhang2020-wp}\\
        Log-cluster metric & {} & \cite{Goncalves2020}\\
        Support coverage metric & {} & \cite{Goncalves2020}\\
        Time-lagged correlations and covariates & {} & \cite{Fisher2019,walsh2020generating}\\
        Latent Space Representation (LSR) & {} & \cite{yan2020generating}\\
        Distribution of Jaccard similarity & {} & \cite{ozyigit2020generation}\\
        \bottomrule
    \end{tabular}
\end{table}

\begin{table}
        
        
        \caption{Metrics of data realism employing methods and measures based on evaluating the utility of the synthetic data on practical tasks.}\label{tab:aug-metrics}
        
        \begin{tabular}{@{} p{0.2\textwidth} p{0.2\textwidth} p{0.2\textwidth} @{}} \toprule
        Metric & Description & References\\ \midrule
        
        \multicolumn{3}{Y}{\textbf{Data utility metrics}}\\ \midrule
        
        Dimension-wise prediction (DWP) & Each variable is in turn chosen as the prediction target label and the remaining as features. Two predictors are trained to predict the label, one from the synthetic data and another from a portion of the real data. Their performance is compared on the left out real data.  & \cite{choi2017generating,Camino2018-re,Goncalves2020,yan2020generating}\\[20pt]
        
        Association Rule Mining (ARM) & & \cite{Baowaly2019,Bae2020,yan2020generating}\\[20pt]
        
        Discriminative Siamese architecture & & \cite{torfi2019generating}\\[20pt]
        
        Train on synthetic, test on real (TRTS) & Accuracy on real data of some form of predictor trained on synthetic data \cite{Beaulieu-Jones2019-ct}. Correlation between important features (RF) and model coefficients (LR and SVM) \cite{Beaulieu-Jones2019-ct}. & \cite{esteban2017real,Xu2019-ay,Yoon2018-dm,chin2019generation}\\
        
        Accuracy on synthetic data of some form of predictor trained on real data & & \cite{Bae2020}\\
        
        Forward prediction accuracy of conditional generative model &
        Models trained to make forward predictions from past observations or from real data transformed with a known function can simply be evaluated for accuracy. & \cite{Xiao2018-aj,mcdermott2018semi,yoon2018gain,Yang_2019b}\\
        

        \multicolumn{3}{Y}{\textbf{Applied utility metrics}}\\ \midrule

        
        Data augmentation & A predictor is trained on a combination dataset of real and synthetic data and performance is compared with the same predictor trained on real data alone. & \cite{Yoon2018-mo}\\
        
        Predictor augmentation & The trained generative model is incorporated into a predictor's activation function by generating an ensemble of proximate data points for each instance, thereby improving generalization. & \cite{Che_2017}\\
        
        \bottomrule
        
        \end{tabular}
\end{table}

\subsection{Alternative evaluation}
In their publications, Yale et al. propose refreshing approaches to evaluating the utility of synthetic data. For example, they organized a hack-a-thon type challenge. During the event, students were tasked with creating classifiers, while provided only with synthetic data \cite{Yale_2020}. They were then scored on the accuracy of their model in real data. Similarly, in a different evaluation experiment, they attempted (successfully) to recreate published medical papers based on the MIMIC dataset using only data generated from their model HealthGAN. The implications of these results for exploratory data analysis, reproducibility experiments in cases where data cannot be distributed and more generally education in health-related scientific training are glaring. In a subsequent paper, the authors evaluate the performance of their model against traditional privacy preservation methods by using the trained discriminator component of HealthGAN to d
        
        \section{Privacy Preservation}
To evaluate the risk of reidentification of synthetic data in the publications included, empirical analyses of privacy preservation are conducted according to the definitions of Membership Inference (MIA), Attribute Disclosure (AD)  \cite{choi2017generating,Goncalves2020,yan2020generating} and Reproduction rate \cite{Zhang2020-wp}. Cosine similarities between pairs of samples are also employed \cite{torfi2019generating}. All studies report low success rates for these types of attacks, while there is little effect from the sample size. Broadly, an MIA attack aims to determine if a particular record was used to train a machine learning model \cite{chen2019ganleaks}. There is no canonical process by which an attack is conducted, nor specification of the data assets initially in possession of the attacker. For a comprehensive taxonomy of MIA against GANs, refer to the suitably titled publication by Chen et al. in which medGAN was subjected to a number of trials.\par
In black-box and white-box type attacks, including the LOGAN \cite{hayes2017logan} method, medGAN performed considerably better than WGAN-GP \cite{gulrajani2017improved}, the algorithm which served as basis for improvements to medGAN in publications discussed in Section 3.1. Overall, the authors note that releasing the full model poses a high risk of privacy breaches and that smaller training sets (under 10k) also lead to a higher risk.\par  
AD is defined as the risk of an attacker correctly infering unknown attributes of a patient's record, given a number of known attributes. Goncalves et al. evaluated MC-medGAN against multiple non-adversarial generative models in a variety of privacy compromising attacks, including AD, obtaining inconsistent results for MC-medGAN \cite{Goncalves2020}. While this is not mentioned by the authors, multiple results reported in the publication point to the fact that the GAN was not properly trained or suffered mode-collapse.\par
Numerous attempts have been made to confer traditional privacy guarantees that deteriorate data, such as differentially-private stochastic gradient descent. By limiting the gradient amplitude at each step and adding random noise, AC-GAN could produce useful data witth $\epsilon=3.5$ and $\delta<10^{-5}$ according to the definition of differential privacy \cite{Beaulieu-Jones2019-ct, esteban2017real,chincheong2020generation}. \cite{BaeAnomiGAN2020}. Uniquely, Bae et al. ensure privacy with a probabilistic scheme that ensure indistinguishably, but also maximizes utility. Specifically, a multiplicative perturbation by random orthogonal matrices with input entries of $k × m$ medical records and a second second discriminator in the form of a pretrained  predictor \cite{Bae2020}. Means to confer privacy guarantees on synthetic data generated by GANs are being actively researched in a variety of fields, many of which are a priori readily applicable to health data. At this stage, however, contradictory results have between obtained where the statistical fidelity of the synthetic seemed to be preserved, but utility-based measures based on a classification were degraded by incorporating DP.\par
\subsection{The status of fully synthetic in regards to current privacy regulations}
It seems intuitively possible that the artificial nature of synthetic data essentially prevents associations with real patients, however the question is never directly addressed in the publications. An extensive Stanford Technological Review legal analysis of synthetic data concluded that laws and regulations should not treat synthetic data indiscriminately from traditional privacy preservation methods \cite{bellovin2019privacy}. They state that current privacy statutes either outweigh or downplay the potential for synthetic data to leak secrets by implicitly including it as the equivalent of anynonymization. 
\subsection{GAN-centric approach to privacy}
Some have put forward the notion that preventing overfitting and preserving privacy may not be conflicting goals \cite{Wu2019-ui,Mukherjee2019-vu}. In privGAN, Mukherjee et al., an adversary is introduced, forcing the generator to produce samples that minimize the risk of MIA attack, in addition to cheating the discriminator. The combination of both goals has the explicit effect of preventing overfitting, and their algorithm produces samples of similar quality to non-private GANs.\par
The discordance between the theoretical concepts of DP, which are  based ultimately on infinite samples, and the often insufficient data on which the probability of disclosure is calculated remains deficient. Therefore, Yoon et al. have postulated an intriguing alternative view of privacy \cite{Yoon2020}. They propose to emphasize measuring identifiability of finite patient data, rather than the probabilistic disclosure loss of DP based on unrealistic premises. Simplistically, they define identifiability as the minimum closest distance between any pair of synthetic and real samples. In their implementation, the generator receives both the usual random seed and a real sample as input. This has the effect of mitigating mode collapse, but also of reproducing the real samples. On the other hand, the discriminator is equipped with an additional loss metric based on a measure of similarity between the original sample and the generated one, thus ensuring the tuneable threshold of identifiability is met. Their results on a number of previously discussed evaluation metrics are encouraging.\par
In a similar approach, Yale et al. broke away from the theoretical guarantees of traditional methods with a measure native to GANs. Their proposal is a metric quantifying the loss of privacy, a concept more aligned with the objective of GANs to minimize the loss of data utilit \cite{yale:hal-02160496,p2019}. They point out, quite appropriately, the advantage of concrete measureable values of loss in utility and privacy when making the decision of releasing sensitive data. Briefly, the Nearest Neighbor Adversarial Accuracy measures the loss in privacy based on the difference between two nearest neighbor metrics. The  first component is the proportion of synthetic samples that are closer to any real sample than any pair of real samples. The second component is the reverse operation. In a subsequent paper, HealthGAN evaluated against traditional privacy preservation methods with a variant of the IA based on the nearest neighbor metric. HealthGAN performs considerably better than all other methods, while still maintaining utility on a prediction task .


       



